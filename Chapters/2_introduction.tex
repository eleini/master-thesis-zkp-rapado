\chapter{Introduction}
\citet{kliewer2005optimierung}
\citet{semesterproject}
\section{Motivation}
A modern civil aircraft consists of approximately three million parts, of which thousands of components must be maintained, repaired, and overhauled (MRO) throughout its life-cycle. These MRO events have to be documented. If aircraft and accompanying components demonstrate safe technical conditions, aviation authorities declare them airworthy; hence, they allow them to operate \citep{FornaconFrank}. Considering there are 25,000 commercial aircraft in operation and approximately 20,000 suppliers in the industry, there are hundreds of millions of events to be documented and verified \citep{mroBCservices1}.

The aircraft part life-cycle starts with initial manufacturing documentation and continues with buying, selling, leasing, repair and overhaul documentation, as well as disposal documentation \citep{FornaconFrank, mroBCservices}. To this day, back-to-birth documentation of aircraft and components is still largely paper-based and lacks digital process automation \citep{efthymiou}. Transparency and traceability are compromised through disconnected, scattered MRO event documentation entries across various enterprise resource planning systems operated by the industry's participants \citep{FornaconFrank, mroBCservices1}. Aircraft components without verified back-to-birth documentation history have no value, i.e., more sustainable second-hand trading of parts is not possible. This overall situation has an immediate effect on the verification of aircraft parts documentation by aviation authorities: verifying for airworthiness or traceability in the event of an accident is a cumbersome, inquiry-based process, which leaves room for withholding of information and counterfeit parts \citep{planecrash, efthymiou}.

\textit{RAPADO} is a research project that is driving digitalization in the aviation industry by creating an industry standard for seamless documentation and verification of aircraft spare parts. The research is being conducted by the Information Systems Department at Freie Universit{/"a}t Berlin in collaboration with project partner Opremic Solutions GmbH and is being funded by the German Federal Ministry of Economic Affairs and Climate Action. Preliminary work at the department resulted in further research in the application of blockchain technology and encryption methods with the aim of targeting the requirements of documentation storage and traceability. Previously, a DApp was developed, where aircraft part certificates can be stored in a distributed file storage (IPFS) and traced via the file hash, which is persisted on an Ethereum blockchain.

\subsection{Outline of Contribution}
The aviation industry is characterized by high competition and mistrust among the stakeholders \citep{Chatzi2019TDoC}, which creates conflicting expectations when pursuing the goal to digitalize and automate its processes. Aircraft parts documentation history should always be verifiable and tamper-proof, while preserving data confidentiality and protecting the identity and ownership of industry participants \citep{Wickboldt2019BlockchainFW}. Facilitating such a verification process without revealing sensitive information can be achieved through zero-knowledge proof (ZKP) algorithms.

This master's thesis focuses on the requirements of verification and information confidentiality within project RAPADO. It extends previous findings by reviewing current literature on the topic of zero-knowledge proof protocols and practical examples. Additionally, suitable use cases and their implementation are investigated. The research question is formulated as follows:

\textit{Which requirements and use cases for zero-knowledge proof protocols can be discovered in the RAPADO project and how can they be implemented?} 
%Adam-Air-Flug 574 Beispiel
%formulate topic of master thesis, research question(s)
\begin{comment}


c)research question
d)Relevanz/Contribution: what is my contribution to the topic
-->1)artifact(s)
\end{comment}
\begin{comment}
1.1 Motivation = Problem

a)how do I start a problem motivation? introduce the domain! -to give readers some time to breathe, especially 
when they're coming from a different research area

b)introduce the problem
- if I talk about the problem, I never talk abot the solution
- start with zero, for readers that have almost no knowledge in the field, write it in an understandable way--> get more detailed
to the end
-give examples/use cases (not a must)-->something the readers has heard of, as easy as possible but as complex as possible 
to later reference it and build it up
-why is it important to solve this problem?

c) (Research Question) for us because it's a seminar paper: formal problem definition!
2 properties of RQ and problem defition: exact and complete (what the paper is about)
what's the difference between FPD (formal problem definition) and RQ tho?
\end{comment}

\begin{figure}
	\centering
		\includegraphics[width=0.7\textwidth]{./Pictures/bsp1.png}
	\caption{Beispiel 1 zum Einf{/"u}gen einer Grafik}
	\label{fig:bsp1}
\end{figure}

\section{Outline of Contribution}
\begin{comment}
a)what is the aim of the thesis
b)outline of contribution (what is your contribution to the problem? -->solve a problem, give an overview,
implement something)
in technical fields, there are two parts: 
1) artifact (software, knowledge etc.) 
2) methodological approach/reasoning on the way to the artifact-->FOR US: SYTEMATIC LITERATURE REVIEW (our only methodology)
\end{comment}

\section{Structure}
\begin{comment}
ALWAYS START WITH THIS SENTENCE: "This paper/article/etc. is structured as follows." (kein doppelpunkt)
"In Section 2, related work of XYZ is introduced.."
\end{comment}


