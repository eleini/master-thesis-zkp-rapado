\chapter{Introduction}
\section{Motivation}
A modern civil aircraft consists of approximately three million parts, of which thousands of components must be maintained, repaired, and overhauled (MRO) throughout its life cycle. These MRO events have to be documented. If aircraft and accompanying components demonstrate safe technical conditions, aviation authorities declare them airworthy and allow them to operate. Considering there are 25,000 commercial aircraft in operation and approximately 20,000 suppliers in the industry, hundreds of millions of events must be documented and verified \citep{mroBCservices1}.

The aviation industry is characterized by high competition and mistrust among the stakeholders \citep{Chatzi2019TDoC}, creating conflicting expectations when digitalizing and automating its processes. The aircraft part lifecycle starts with initial manufacturing documentation and continues with buying, selling, leasing, repair, overhaul, and disposal documentation. To this day, back-to-birth documentation of aircraft and components is still largely paper-based and lacks digital process automation \citep{efthymiou}. Transparency and traceability are compromised through disconnected, scattered MRO event documentation entries across various enterprise resource planning systems operated by the industry's participants \citep{mroBCservices1}. Aircraft components without verified back-to-birth documentation history have no value, i.e., more sustainable secondary parts trading is impossible. This overall situation has an immediate effect on the verification of aircraft parts documentation by aviation authorities: verifying for airworthiness and traceability in the event of an accident is a cumbersome, inquiry-based process, which leaves room for withholding of information and counterfeit parts \citep{planecrash}. 
The research project RAPADO drives digitalization in the aviation industry by creating an industry standard for seamless documentation and verification of aircraft spare parts. One of the research goals is to accumulate knowledge about blockchain technology and investigate its applicability to the MRO industry. Research is conducted by the Information Systems Department at Freie Universit{\"a}t Berlin in collaboration with project partner Opremic Solutions GmbH.

Current work on blockchain in the MRO industry focuses on empirical analyses of the benefits of the technology in general and the willingness to adopt \citep{efthymiou}. At the same time, further technical understanding and prototypical implementations must be performed. Project-related work resulted in further research into the application of blockchain technology to target the requirements of document storage and traceability, excluding the scope of state-of-the-art cryptography. Previously, a blockchain-based file storage system was proposed, prototypically developed as a decentralized application (DApp), storing aircraft part certificates in a distributed file storage system. Certificate data is traceable through a file hash, which is persisted on-chain. This preliminary work concludes with zero-knowledge proofs as the next step in project-related research \citep{ZedelJ}. Aircraft parts documentation history should always be verifiable and tamper-proof while preserving data confidentiality and protecting the identity and ownership of industry participants \citep{Wickboldt2019BlockchainFW}. The aspect of decentralized verification with the requirement of data privacy preservation still needs to be addressed.

Zero-knowledge proofs can facilitate such a verification process without revealing sensitive information. Due to the possibility of combining confidentiality and transparency, and scaling blockchain throughput, interest in zero-knowledge proofs increased. However, their potential application in the aviation industry needs further assessment. This master thesis focuses on verification and information confidentiality requirements for aircraft spare parts documentation. Suitable use cases and their implementation through zero-knowledge proofs are investigated. The research question is formulated as follows:

\begin{center}
\textit{How can zero-knowledge proofs be utilized to satisfy requirements and implement identified use cases for the aircraft spare parts documentation?}  
\end{center}

\section{Contribution Outline}
In this master thesis, the following aspects of contribution are outlined. The topic of zero-knowledge proofs is complex and needs to be discussed separately, as concluded in preliminary research. A systematic literature review, based on the methodology in \citet{vomBrockeJan2019TDgs} and \citet{Webster2002AnalyzingTP}, surveys current research results and accumulates expertise in the field. As a more practical result, a ZKP protocol, i.e., Groth16, is applied step by step to provide an overview based on a simple polynomial example. Specific details for understanding the protocol support the mathematical example, forming a reference document acting as the first artifact of the thesis.

Current research requirements in the project RAPADO are revisited to derive implementation requirements and select use cases. Taking up the use case of MRO data attestation and verification, the developed zero-knowledge decentralized application (zk-DApp) is a proposal to balance transparency and data confidentiality. The zk-DApp is evaluated based on performance aspects and feedback within the project.

For all existing software implementations within the project, high-level assumptions about a digital data structure for aircraft spare part documentation were necessary to continue with prototype development. The question of a practical, blockchain-compatible data structure was not yet addressed and reoccurred as a feedback point during discussions in the project. Motivated by the use case of authenticity checks of spare part certificates and as an evaluation result of the zk-DApp mentioned above, a ZKP and Merkle-tree-based data structure based on \citet{sedlemeirgrenenergy} is introduced. Concluding with the data structure proposal, a direction of future research is suggested, considering existing software artifacts and results.

\section{Structure}
This master thesis is structured as follows. Chapter 2 summarizes related work, separated into project-related work, zero-knowledge proof topic-related work, and related systems. Chapter 3 describes the execution of the systematic literature review methodology, the artifacts' implementation approach, and summarizes key findings. Chapter 4 presents the outcomes of the systematic literature review, structured from the perspective of historical and designing-oriented classification of zero-knowledge proofs. Chapter 5 summarizes implementations and results. First, the requirements are summarized. Second, the three artifacts are described, each satisfying a requirement. The first artifact is the example calculation and step-wise computation of the Groth16 algorithm. The second artifact is the zk-DApp for MRO data attestation and verification. Subsequently, the zk-DApp is evaluated, resulting in the third artifact's design. The third artifact is the fraud-preventive zero-knowledge data structure for data authenticity and integrity verification. Chapter 6 concludes this master thesis by summarizing key findings, describing identified limitations, and providing a future outlook for research.

