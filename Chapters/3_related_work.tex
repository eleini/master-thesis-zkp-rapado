\chapter{Related Work}
\begin{comment}
-knowledge base I build upon
- related papers can be: content related, methodology, technical -->structure accordingly

(in a thesis, it is mandatory to have Related Systems, when you implement an application)
- Sedlmeier, V{/"u}lter, Str{/"u}ker (2021): trading green energy certificates: same problem of verification vs privacy, Implementation architecture based on zk RollUps, very good reasoning about requirements (same as Rapado), maybe a good starting point to create an architecture as second artifact for rapado? architecture to verify information about certificates
\end{comment}
Preliminary research at the Department of Information Systems focuses on the architecture of suitable blockchain-based platforms for aviation industry MRO documentation. \citet{WickboldtMeiseKliewer} proposes a framework to use a private blockchain-based architecture in HyperLedger Fabric, whereby node registration is managed through trusted Membership Service Providers (MSP). This first approach resulted from initial core requirements of data persistence, selective data access, data integrity, and transparency of back-to-birth documentation histories \citep{WickboldtClemens2018BzdD}. Subsequently, these core requirements were further specified through extensive information exchange with previously identified stakeholder groups of airline companies, MRO full-service providers, and MRO parts merchants \citep{ZedelJ}. Further research proposed a public blockchain-based platform with smart contracts and a decentralized file storage system \citep{semesterproject, ZedelJ}. MRO documentation is stored in the decentralized file storage system, which produces a unique file hash. It also identifies the document that is implemented as a non-fungible token (NFT) in a smart contract. Following the identified process, the validation status set by aviation authorities is captured. In a DApp software artifact, aviation authorities get full document access using threshold encryption. They receive their key shares from the smart contract and can decrypt them using their private key. By combining their key shares, the corresponding document can be decrypted, accessed, and used for verification \citep{semesterproject}. Hence, access management is not central because various nodes in the network have to combine their key shares. However, once access is granted, all data is exposed. Documents contain competitive information and must be treated confidentially. This led to further research on how to verify MRO documentation without fully exposing sensitive information. Zero-knowledge proofs were identified as promising technology to meet the requirement of data confidentiality during the verification of MRO documents on a public blockchain-based platform \citep{ZedelJ}. This master's thesis extends previous research by focusing on the topic of ZKP. Use cases are examined and implemented with regards to business processes and requirements developed within project RAPADO.
%content related, methodology, technical -->structure accordingly
%on chain computations are expensive (see project semester), ZKPs are powerful off chain solutions for verification processes

%look at some review paper in WOS and tease them
%present previous work (Wickboldt, Zedel, Kliewer, semester project)