\chapter{Related Work}
Zero-knowledge argument systems are algorithms resulting from specific designs of proof systems combined with cryptographic primitives and mathematical tools. This chapter summarizes content-related and project-related work and systems associated with this master thesis. Revisiting the project RAPADO, current and previous research and implementation efforts at the Department of Information Systems at Freie Universit{\"a}t Berlin are described. The selective systematic literature review in Chapter 5 provides a primary knowledge survey and references for further information. The results of this master thesis are a step-by-step example computation of the Groth16 protocol, a zero-knowledge \acrshort{dapp} software artifact, and a data structure conceptual artifact. The related systems utilized for artifact development are presented in this chapter.

\section{Topic-related Work}
In \citet{Thaler}, an in-depth survey is provided on the history of verifying computation and content classification of probabilistic proofs. Furthermore, more practical argument systems are described, and their composition is studied, while the reader is provisioned with a bird's eye view of the proof systems and cryptographic primitives. First, interactive, multi-prover interactive, interactive oracle, probabilistically checkable proof systems, and variants are described in more detail. Second, concrete example proof systems resulting from historical breakthroughs in the field are highlighted. Lastly, relevant cryptographic primitives and mathematical tools are introduced that enable non-interactivity, efficiency, and zero knowledge. The reader is presented with a taxonomy of succinct non-interactive arguments of knowledge (\acrshort{snark}s).

The work of \citet{chen2022review} surveys \acrshort{zksnark}s from a technical perspective. First, the historical development of \acrshort{zksnark}s is described. Second, the first state-of-the-art \acrshort{zksnark}s, the Pinocchio protocol, is analyzed in detail and compared to its successor Groth16. Third, two prominent use cases of \acrshort{zksnark}s are highlighted, financial and rollups-based applications. Lastly, novel circuits are introduced and applied in private auctions and decentralized card games, and the implementation code is provisioned. A future research outlook is portrayed by introducing the current research status on zero-knowledge, scalable, transparent arguments of knowledge (\acrshort{zkstark}s) and recursive \acrshort{zksnark}s.

A survey on verifiable computation is provided by \citet{Ahmad}, focusing on the chronological summary of theoretical and practical advances. Approaches are summarized according to their functionalities and analyzed according to their contributions. The authors comment on open challenges in verifying computation and give an outlook for research efforts. 

Zk-SNARKs are the main focus in \citet{NitulescuGentleIntroSNARKs}. First, properties are defined, and tools for designing \acrshort{zksnark}s are described. Second, \acrshort{snark}s from probabilistically checkable proofs, quadratic arithmetic programs, linear interactive, polynomial interactive oracle proof systems, and variants are introduced, highlighting more practical use cases. 

\section{Project-related Work}

Project RAPADO is part of the aviation research program in the German Federal Ministry of Economic Affairs and Climate Action, which, among other things, supports research and development of disruptive technologies to be implemented in the aviation industry within the next decade. It is funded by the German Federal Ministry of Economic Affairs and Climate Action and the German Aerospace Center (\acrshort{dlr}) within the Federal Aviation Research Programme (LuFo Klima VI-1, European Commission State Aid SA.55829 (2019/N)). This project's primary goal is to research seamless, complete, and safe documentation and certification of aircraft spare parts materials using current developments in blockchain technology. The common practice of reusing, repairing, and trading aircraft spare parts is strengthened and permanently transformed. The \acrshort{mro} industry operates in an error-prone and non-digital manner: If a spare part is repaired and ready to be returned, the corresponding certificates and receipts are sent via mail, i.e., purchasing complex spare parts results in paper-based documentation being delivered on pallets. Hence, \acrshort{mro} providers are not motivated to reuse spare parts and instead buy new ones to prevent risks and liabilities due to incomplete documentation. Spare parts are only used for civil transportation with complete documentation history. The project consortium, consisting of representatives from Freie Universit{\"a}t Berlin and industry partner Opremic Solutions GmbH, aims at creating a digital process to reduce the production of new spare parts, enable trade, and provide increased security through persistent documentation while contributing to the funding target of productive and efficient aerospace. The industry standard is distributed as open-source, driven by a critical mass of users and the International Air Transportation Association (IATA) as a partner.

Preliminary research at the Department of Information Systems was focused on the architecture of suitable blockchain-based platforms for aviation industry \acrshort{mro} documentation. \citet{WickboldtMeiseKliewer} proposes a framework to use a private blockchain-based architecture in HyperLedger Fabric, whereby node registration is managed through trusted Membership Service Providers (MSP). This first approach resulted from initial core requirements of data persistence, selective data access, data integrity, and transparency of back-to-birth documentation histories \citep{WickboldtClemens2018BzdD}. Subsequently, these core requirements were further specified through extensive information exchange with previously identified stakeholder groups of airline companies, \acrshort{mro} full-service providers, and \acrshort{mro} parts merchants \citep{ZedelJ}. Further research proposed a public blockchain-based platform with smart contracts and a decentralized file storage system \citep{ZedelJ}. \acrshort{mro} documentation is stored in the decentralized file storage system, which produces a unique file hash. Documents are implemented as a non-fungible token (NFT) in a smart contract.

Following the identified process, the validation status set by aviation authorities is captured. In a \acrshort{dapp} software artifact, aviation authorities get full document access using threshold encryption. They receive their key shares from the smart contract and can decrypt using their private key. Combining their key shares allows the corresponding document to be decrypted, accessed, and verified. Access management is separate because various nodes in the network have to combine their key shares. However, once access is granted, all data is exposed. Documents contain competitive information and must be treated confidentially. Zero-knowledge proofs were identified as promising technology to meet data confidentiality requirements while verifying \acrshort{mro} documents on a blockchain-based platform \citep{ZedelJ}. This master thesis extends previous research by focusing on zero-knowledge proofs and applications for sensitive data verification.

The current project status and outlook for the remaining 16 months of the project run time were discussed in the workshop on May 8, 2023, in interaction with the \acrshort{dlr}. The following project parties attended: experts from the aviation industry on the business product side (Opremic GmbH), experts from the aviation industry on the technical product side and data privacy aspects (Opremic GmbH), researchers on the topic of mobility systems, blockchain, and zero-knowledge proofs (Department of Information Systems at Freie Universit{\"a}t Berlin). 

In workshop discussions, it is suggested that adopting \acrshort{dlt}-based solutions for \acrshort{mro} documentation depends on full compliance with requirements set by the operating model in the aviation industry, whereby the main compatibility aspects combine the benefits of data persistence and privacy while capturing spare part documentation. Data persistence describes the constant accessibility of spare part documentation and \acrshort{mro} event history. It is often called tamper-proof data storage and retention in an industry context. Decentralized networks increase data availability, which needs further assessment regarding \acrshort{dlt}-based solutions in the project. Data availability is seen as a result of a high number of nodes operated in the network, which also bears the risk of redundancy and efficiency loss.

Further research about off-chain possibilities and the degree of decentralization can help counteract this risk. The research topic of zero-knowledge proofs is identified as an opportunity during the project, whereby this thesis serves as a knowledge accumulation and implementation proposal for the project requirements of data privacy compliance and transparent validation of \acrshort{mro} spare parts. Furthermore, efficient implementation and consolidation into existing prototypes facilitate potential cost reductions and performance advances. The research project requirements are summarized to achieve data integrity, persistence, and transparency. Through achieving this goal, the RAPADO project contributes to creating an industry standard and paradigm change for the aviation business.

\section{Related Systems}
Based on the smart contract for zkDocs created by \citet{zkdocs}, the \acrshort{zkdapp} for \acrshort{mro} data input verification is implemented and described. For this use case, additional operators and a new schema are added, and changes to the frontend are implemented. Chapter 5.3 describes the implementation. In contrast, the non-zero-knowledge \acrshort{dapp} implementation developed during the RAPADO semester project uses encrypted and decrypted key shares via the secrets.js library, which are combined to view and validate an \acrshort{mro} certificate for a specific part. The focus of the previously developed \acrshort{dapp} is to create a first structure to upload \acrshort{mro} certificates. The validation implementation using only Shamir's secret sharing is vulnerable. The \acrshort{zkdapp} implementation focuses on privacy and transparency requirements for the attestation and verification of specific \acrshort{mro} data and digitization concepts. Requirements are satisfied through examining the technicalities of zero-knowledge proofs and their implementation as per current research. The second prototypical artifact is an architecture proposal for \acrshort{mro} data digitization to preserve data confidentiality while offering requirement-sufficient transparency via zero-knowledge proofs and Merkle trees. A similar first attempt is presented in \citet{sedlemeirgrenenergy} for trading green energy certificates.