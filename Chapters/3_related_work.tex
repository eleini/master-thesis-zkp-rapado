\chapter{Related Work}
\begin{comment}
-knowledge base I build upon
- related papers can be: content related, methodology, technical -->structure accordingly

(in a thesis, it is mandatory to have Related Systems, when you implement an application)
\end{comment}

Zero-knowledge argument systems are algorithms resulting from specific designs of proof systems combined with cryptographic primitives and mathematical tools. This chapter summarizes content-related and project-related work, as well as related systems associated with this master's thesis. Related to the project RAPADO, current and previous research and implementation efforts at the department of information systems at Freie Universit{\"a}t Berlin are described. Related to the selective systematic literature review in chapter 5, main knowledge surveys are presented to be referred to for further information. The results of this master's thesis are a step-by-step example computation of the Groth16 protocol, a zero-knowledge DApp software artifact and a verification mechanism architecture conceptual artifact. The related systems utilized for artifact development are presented in this chapter.

\subsubsection{Project-related work}
Preliminary research at the Department of Information Systems focuses on the architecture of suitable blockchain-based platforms for aviation industry MRO documentation. \citet{WickboldtMeiseKliewer} proposes a framework to use a private blockchain-based architecture in HyperLedger Fabric, whereby node registration is managed through trusted Membership Service Providers (MSP). This first approach resulted from initial core requirements of data persistence, selective data access, data integrity, and transparency of back-to-birth documentation histories \citep{WickboldtClemens2018BzdD}. Subsequently, these core requirements were further specified through extensive information exchange with previously identified stakeholder groups of airline companies, MRO full-service providers, and MRO parts merchants \citep{ZedelJ}. Further research proposed a public blockchain-based platform with smart contracts and a decentralized file storage system \citep{semesterproject, ZedelJ}. MRO documentation is stored in the decentralized file storage system, which produces a unique file hash. It also identifies the document that is implemented as a non-fungible token (NFT) in a smart contract. Following the identified process, the validation status set by aviation authorities is captured. In a DApp software artifact, aviation authorities get full document access using threshold encryption. They receive their key shares from the smart contract and can decrypt them using their private key. By combining their key shares, the corresponding document can be decrypted, accessed, and used for verification \citep{semesterproject}. Hence, access management is not central because various nodes in the network have to combine their key shares. However, once access is granted, all data is exposed. Documents contain competitive information and must be treated confidentially. This led to further research on how to verify MRO documentation without fully exposing sensitive information. Zero-knowledge proofs were identified as promising technology to meet the requirement of data confidentiality during the verification of MRO documents on a public blockchain-based platform \citep{ZedelJ}. This master's thesis extends previous research by focusing on the topic of zero-knowledge proofs and applications for sensitive data verification.

\subsubsection{Content-related work}
In \citet{Thaler}, a in-depth survey is provided on the history of verifying computation and content classification of probabilistic proofs. Furthermore, more practical argument systems are described and their composition is studied, while provisioning the reader with a bird's eye view on the proof systems and cryptographic primitives needed to understand those. First, interactive, multi prover interactive, interactive oracle, probabilistically checkable proof systems and variants are described in more detail, introducing concrete examples proof systems from historical breakthrough in the beginning of research towards zero-knowledge argument systems. Second, relevant cryptographic primitives and mathematical tools are introduced, that enable non-interactivity, efficiency and zero-knowledge. Lastly, the reader is presented with a taxonomy of succinct non-interactive arguments of knowledge (SNARKs).

The current and compact work of \citet{chen2022review} surveys zk-SNARKs from a technical perspective. First, the historical development of zk-SNARKs is described. Second, the first state-of-the-art zk-SNARKs, the Pinocchio protocol is analyzed in great detail, and compared to its successor Groth16. Third, two main use cases of zk-SNARKs are highlighted, financial and rollup applications. Lastly, novel circuits are introduced, applied in the domain of private auctions and decentralized card games, and the implementation code is provisioned. A future research outlook is portrayed by introducing the current research status on zk-STARKs and recursive SNARKs.

A shorter survey on verifiable computation is provided by \citet{Ahmad}, whereby the chronological summary of theoretical and practical advances are at focus. Approaches are summarized according to their functionalities and analyzed according to their contributions. The authors provide comments on open challenges in verifying computation and give a future outlook for research efforts. 

Zk-SNARKs are the main focus in \citet{NitulescuGentleIntroSNARKs}. First, properties are defined and tools for designing zk-SNARKs are described. Second, SNARKs from probabilistically checkable proofs, quadratic arithmetic porgrams, linear interactive and polynomial interactive oracle proof systems and variants are introduced, whereby more practical use cases are highlighted. 


\subsubsection{Related systems}
%on chain computations are expensive (see project semester), ZKPs are powerful off chain solutions for verification processes
%zkDocs smart contract and framework used for MVP
%- Sedlmeier, V{/"u}lter, Str{/"u}ker (2021): trading green energy certificates: same problem of verification vs privacy, Implementation architecture based on zk RollUps, very good reasoning about requirements (same as Rapado), maybe a good starting point to create an architecture as second artifact for rapado? architecture to verify information about certificates