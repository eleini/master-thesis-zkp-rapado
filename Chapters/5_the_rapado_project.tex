\chapter{The RAPADO Project}
\section{Project Introduction}
Project RAPADO is part of the aviation research program in the German Federal Ministry of Economic Affairs and Climate Action, which, inter alia, supports research and development of disruptive technologies to be implemented in the aviation industry within the next 20 to 30 years. Main goal of this project is conducting research on seamless, complete, and safe documentation and certification of aircraft spare parts materials using current developments in blockchain technology. The common practice to reuse, repair and trade aircraft spare parts is strengthened, as well as safely and permanently transformed. The MRO industry operates in an error-prone and non-digital manner: If a spare part is repaired and ready to be returned, the corresponding certificates and receipts are sent via mail, i.e., purchasing complex spare parts results in paper-based documentation being delivered on pallets. Hence, MRO providers are not motivated to reuse spare parts and rather buy new to prevent risks and liabilities due to incomplete documentation, because spare parts are not of any use for civil transportation without complete documentation history. The project consortium, consisting of representatives from Freie Universit{\"a}t Berlin and industry partner Opremic Solutions GmbH, aims at creating a digital process to reduce production of new spare parts, enable trade, and provide an increased security through persistent documentation, while contributing to the funding target of productive and efficient aerospace. The industry standard is distributed as open-source, driven by critical mass of users and the International Air Transportation Association (IATA) as partner.

\section{Research Status}
In consideration of preliminary work of \citet{ZedelJ, Wickboldt2019BlockchainFW, semesterproject} in the project, the project status is reflected on and current project requirements are stated as basis for this thesis' outcomes.

Recent project outcomes suggest that the adoption of DLT-based solutions for MRO documentation is dependent on full compliance of requirements set by the operating model in the aviation industry, whereby the main compatibility aspects are combining the benefits of data persistence and privacy while capturing spare part documentation. Private decentralized networks were suggested \citep{Wickboldt2019BlockchainFW}. Data persistence describes the constant accessibility of spare part documentation and MRO event history. In an industry context, it is often referred to as tamper-proof data storage and retention. Decentralized networks increase data availability, which needs further assessment with regards to DLT-based solutions in the project. High availability is seen as a result of a high number of nodes operated in the network, which also bears the risk of redundancy and efficiency loss. Further research of off-chain possibilities can help counteract this risk.

-look further to semester project intros: 
(i) High speed of the system to not intervene with processes of part usage.
(ii) Consolidation of parts and certificates from different sources.
(iii) Persistence of data while operating in a trust-free and permission-based environment to secure anti-counterfeiting and confidentiality.

next:
-security and privacy
-transparency
- concepts for privacy identified: ZKP as one outcome of previous research for future research
- for this, three main requirements resulted: 1)knowledge accumulation 2)privacy vs transparency 3)digitization and data formats fit for purpose
--> use backings from project literature used also in semester project and jans publication
\begin{comment}
-go thru zedel kliewer and list principles, say that zkps were one solution criteria mentioned already
\section{Project Introduction}
\subsection{Background}
\subsection{Initial Problem Analysis}
\subsection{Previous Work}
\subsection{Reflection}
\section{Current Project Status}
\subsection{Problem Analysis}
\subsection{Design Requirements}
\subsection{Derived Use Cases}

\begin{table}[bp]
	\centering
		\caption{Beispiel 1 zum Einfügen einer Tabelle}
		\begin{tabular}{| c c c |}
		\hline
			&&\\
			Monat & Linie & Minuten\\
			\hline
			\hline
			&&\\
			Jan & U7 & 10 \\
			Feb & U9 & 12 \\
			Mär & U9 & 20 \\
			\hline
		\end{tabular}

	\label{tab:Beispiel1}
\end{table}


\begin{figure}[tp]
	\centering
		\includegraphics[width=0.50\textwidth]{./Bilder/bsp2.png}
	\caption{Beispiel 2 zum Einfügen einer Grafik}
	\label{fig:bsp2}
\end{figure}
\end{comment}
