\chapter{The RAPADO Project}
\section{Project Introduction}
Project RAPADO is part of the aviation research program in the German Federal Ministry of Economic Affairs and Climate Action, which, inter alia, supports research and development of disruptive technologies to be implemented in the aviation industry within the next 20 to 30 years. Main goal of this project is conducting research on seamless, complete, and safe documentation and certification of aircraft spare parts materials using current developments in blockchain technology. The common practice to reuse, repair and trade aircraft spare parts is strengthened, as well as safely and permanently transformed. The MRO industry operates in an error-prone and non-digital manner: If a spare part is repaired and ready to be returned, the corresponding certificates and receipts are sent via mail, i.e., purchasing complex spare parts results in paper-based documentation being delivered on pallets. Hence, MRO providers are not motivated to reuse spare parts and rather buy new to prevent risks and liabilities due to incomplete documentation, because spare parts are not of any use for civil transportation without complete documentation history. The project consortium, consisting of representatives from Freie Universit{\"a}t Berlin and industry partner Opremic Solutions GmbH, aims at creating a
\begin{comment}
Ziel von RAPADO ist die Erforschung einer lückenlosen und sicheren Dokumentation und Zertifizierung von Aircraft Surplus Materialien unter Anwendung der Distributed-Ledger-Technologie. Die in der Luftfahrtindustrie gängige Praxis, Flugzeugersatzteile wiederzuverwenden, zu reparieren und zu handeln, wird nachhaltig transformiert und sicher. Die MRO-Branche besteht aus Airlines, Dachverbänden, Werkstätten, Brokern, etc. Die Branche arbeitet wenig digital – wird ein Bauteil repariert,

RAPADO-Verbund, bestehend aus Opremic solutions GmbH und der Freien Universität Berlin, schafft einen geschlossenen, digitalen Prozess, mit dem die Produktion von Teilen reduziert, der Markt für risikobehaftete Ersatzteile geöffnet und höhere Sicherheit durch lückenlose, nicht manipulierbare Dokumentation erreicht wird, was zum Förderziel leistungsfähige und effiziente Luftfahrt beiträgt. Es wird ein Industriestandard geschaffen und als Open Source verbreitet, was durch die kritische Masse an Nutzern und durch den Partner IATA als zentrales Bindeglied aller Akteure der Branche vorangetrieben wird. Im RAPADO Projekt werden Rollenmodelle definiert und plattformbasierte Geschäftsmodelle untersucht. Eine Marchbarkeitsstudie mit MRO-Smarthub-Integration erlaubt eine technische sowie wirtschaftliche Evaluation des neuen Standards.

\section{Project Introduction}
\subsection{Background}
\subsection{Initial Problem Analysis}
\subsection{Project Requirements}
\subsection{Previous Work}
\subsection{Reflection}
\section{Current Project Status}
\subsection{Problem Analysis}
\subsection{Design Requirements}
\subsection{Derived Use Cases}

\begin{table}[bp]
	\centering
		\caption{Beispiel 1 zum Einfügen einer Tabelle}
		\begin{tabular}{| c c c |}
		\hline
			&&\\
			Monat & Linie & Minuten\\
			\hline
			\hline
			&&\\
			Jan & U7 & 10 \\
			Feb & U9 & 12 \\
			Mär & U9 & 20 \\
			\hline
		\end{tabular}

	\label{tab:Beispiel1}
\end{table}


\begin{figure}[tp]
	\centering
		\includegraphics[width=0.50\textwidth]{./Bilder/bsp2.png}
	\caption{Beispiel 2 zum Einfügen einer Grafik}
	\label{fig:bsp2}
\end{figure}
\end{comment}
