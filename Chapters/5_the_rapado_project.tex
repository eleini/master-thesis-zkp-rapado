\chapter{The RAPADO Project}
\section{Project Introduction}
\subsection{Background}
\subsection{Initial Problem Analysis}
\subsection{Project Requirements}
\subsection{Previous Work}
\subsection{Reflection}
\section{Current Project Status}
\subsection{Problem Analysis}
\subsection{Design Requirements}
\subsection{Derived Use Cases}

\begin{comment}

\cite{kliewer2005optimierung}

\begin{table}[bp]
	\centering
		\caption{Beispiel 1 zum Einfügen einer Tabelle}
		\begin{tabular}{| c c c |}
		\hline
			&&\\
			Monat & Linie & Minuten\\
			\hline
			\hline
			&&\\
			Jan & U7 & 10 \\
			Feb & U9 & 12 \\
			Mär & U9 & 20 \\
			\hline
		\end{tabular}

	\label{tab:Beispiel1}
\end{table}


\begin{figure}[tp]
	\centering
		\includegraphics[width=0.50\textwidth]{./Bilder/bsp2.png}
	\caption{Beispiel 2 zum Einfügen einer Grafik}
	\label{fig:bsp2}
\end{figure}
\end{comment}
