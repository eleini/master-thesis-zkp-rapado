\chapter{The RAPADO Project}
\section{Project Introduction}
Project RAPADO is part of the aviation research program in the German Federal Ministry of Economic Affairs and Climate Action, which, inter alia, supports research and development of disruptive technologies to be implemented in the aviation industry within the next 20 to 30 years. Main goal of this project is conducting research on seamless, complete, and safe documentation and certification of aircraft spare parts materials using current developments in blockchain technology. The common practice to reuse, repair and trade aircraft spare parts is strengthened, as well as safely and permanently transformed. The MRO industry operates in an error-prone and non-digital manner: If a spare part is repaired and ready to be returned, the corresponding certificates and receipts are sent via mail, i.e., purchasing complex spare parts results in paper-based documentation being delivered on pallets. Hence, MRO providers are not motivated to reuse spare parts and rather buy new to prevent risks and liabilities due to incomplete documentation, because spare parts are not of any use for civil transportation without complete documentation history. The project consortium, consisting of representatives from Freie Universit{\"a}t Berlin and industry partner Opremic Solutions GmbH, aims at creating a digital process to reduce production of new spare parts, enable trade, and provide an increased security through persistent documentation, while contributing to the funding target of productive and efficient aerospace. The industry standard is distributed as open-source, driven by critical mass of users and the International Air Transportation Association (IATA) as partner.
\begin{comment}
\section{Project Introduction}
\subsection{Background}
\subsection{Initial Problem Analysis}
\subsection{Project Requirements}
\subsection{Previous Work}
\subsection{Reflection}
\section{Current Project Status}
\subsection{Problem Analysis}
\subsection{Design Requirements}
\subsection{Derived Use Cases}

\begin{table}[bp]
	\centering
		\caption{Beispiel 1 zum Einfügen einer Tabelle}
		\begin{tabular}{| c c c |}
		\hline
			&&\\
			Monat & Linie & Minuten\\
			\hline
			\hline
			&&\\
			Jan & U7 & 10 \\
			Feb & U9 & 12 \\
			Mär & U9 & 20 \\
			\hline
		\end{tabular}

	\label{tab:Beispiel1}
\end{table}


\begin{figure}[tp]
	\centering
		\includegraphics[width=0.50\textwidth]{./Bilder/bsp2.png}
	\caption{Beispiel 2 zum Einfügen einer Grafik}
	\label{fig:bsp2}
\end{figure}
\end{comment}
