\chapter{Methodology}
The research concept of the thesis is divided as follows: first, a systematic literature review (SLR) will be conducted according to \cite{HevnerAR2004DSiI, vomBrockeJan2019TDgs, Webster2002AnalyzingTP}. Second, Knowledge from the first part will be applied to the project RAPADO. The implementation part of this master's thesis follows an agile development approach. RAPADO use cases for zero-knowledge proof protocols are investigated, conceptualized, and evaluated, considering aspects found in previously examined literature and preliminary work at the department of information systems at Freie Universit{\"a}t Berlin. The application of acquired technical and theoretical knowledge is at focus, while the use cases implemented can be exchanged in the future during further project research. 

\section{Analysis}
The initial and current status and results of project RAPADO are reflected on. From this analysis, potential use cases and design requirements for the application of zero-knowledge proofs are derived. Following the requirement to accumulate knowledge within the project in order to build expertise in blockchain-based development for the aviation industry, the literature survey focuses on the design of zero-knowledge proofs and theoretical foundations. Opportunities and challenges are displayed by considering practical examples. The second part of the research concept is the implementation of a zero-knowledge application for the use case of MRO data attestation and verification, whereby the demonstration of technical mechanisms is highlighted.

This master's thesis is associated with preliminary work carried out within RAPADO at the department of information systems. Immediate previous research concludes with conceptual solutions and a DApp for aviation industry MRO documentation, centering storage and traceability \citep{ZedelJ, semesterproject}. Hereby, use cases of uploading, storing, and trading aircraft spare parts certificates were given and specific blockchain platform architectures were used. This research extends previous findings. However, it takes a new perspective by further investigating possible use cases for ZKP as methods of automating verification processes, preserving data confidentiality and suggesting suitable data formats. The results are expected to represent the broader research project, i.e., beyond previously used software.

The focus of the SLR are zero-knowledge proofs with the scope definition of classification, opportunities, challenges, evaluation methods, and examples in practice. The final search string is derived from a concept map and the literature found is summarized through a concept matrix.

\setlength{\tabcolsep}{2ex}
\renewcommand{\arraystretch}{1.5}%
\begin{table}[ht]
	\centering
	    \caption{Selected databases and search strings}
		\begin{tabular}{| m{0.1\linewidth} | m{0.8\linewidth}|}
		\hline
		\textbf{Database} & \textbf{Search string} \\ \hline
            Web of Science & TS=("ZKP*" OR "Zero-Knowledge*" OR "zero*knowledge* algorithm*" OR "zero*knowledge*protocol*" OR "zero*knowledge*proof*" OR "zkS*" OR "zk*SNA*" OR "zk*STA*" OR "zk*STO" OR "bulletproof*") \\  \hline
            General search string & (("ZKP*" OR "Zero-Knowledge*" OR "zero*knowledge*algorithm*" OR "zero*knowledge*protocol*" OR "zero*knowledge*proof*" OR "zkS*" OR "zk*SNA*" OR "zk*STA*" OR "zk*STO" OR "bulletproof*") AND ("application*" OR "example*" OR "app*") AND ("carbon*footprint*" OR "*complexity*" OR "evaluation*" OR "cost*" OR "sustainab*" OR "environment*") 
            AND ("challenge*" OR "threat*" OR "problem*")\\ \hline
            ACM Digital \newline Library & TITLE:("ZKP*" OR "Zero-Knowledge*" OR "zero*knowledge*algorithm*" OR "zero*knowledge*protocol*" OR "zero*knowledge*proof*" OR "zkS*" OR "zk*SNA*" OR "zk*STA*" OR "zk*STO" OR "bulletproof*") \newline
            \newline
            (+) ALL:(("ZKP*" OR "Zero-Knowledge*" OR "zero*knowledge*algorithm*" OR "zero*knowledge*protocol*" OR "zero*knowledge*proof*" OR "zkS*" OR "zk*SNA*" OR "zk*STA*" OR "zk*STO" OR "bulletproof*") AND ("application*" OR "example*" OR "app*") AND ("carbon*footprint*" OR "*complexity*" OR "evaluation*" OR "cost*" OR "sustainab*" OR "environment*") AND ("challenge*" OR "threat*" OR "problem*")) \\ \hline 

            Electronic Healthcare & \citet{LuongPark, ZHENG, WangEtAl, Huangetal} \\  \hline 
            Cloud Security & \citet{LiuWangPengXing, Major, Munivel, Kanagamani} \\  \hline 
            Scaling &  \\  \hline 
	\end{tabular}
\label{tab:domains}
\end{table}
\section{Design}
\section{Implementation}
\section{Evaluation}
\begin{comment}
SLR
Analysis of other resources
--> ends with requirements
(how good/bad the solution is)
-quantitative and qualitative evaluation, e.g. Laufzeit und Interviews 
- say what is not in scope
-DO NOT describe some agile method for the sake of having it! Just say it is a rather agile method etc.

I. Systematic literature review according to vom Brocke, Cooper and Webster: 
    1. Definition of Scope
        - classification, examples, challenges and evaluation methods of zero knowledge proof protocols 
        
    2. Conceptualization
        - work with concept map
        - derive at a search string
        
    3. Literature Search and Selection
        - look for review paper first to get good overview about the topic
        a) exclude paper that are too old and have too few citations and/or low impact factor (e.g. 5.5 is high)
        b) exclusion acc. to title and keywords
        c) exclusion acc. to abstract & structure of paper & RQ
        c) exclusion acc. to full text & availability of resource
        
    4. Synthesizing of Literature
        - cluster definitions, examples, drawbacks and evaluation methods (first suggestion can be found in the preliminary agenda)
        - write overview section about ZKP (Chapter 4)
- - - - - - -
How to know if a paper is useful for me?
1.title 2.keywords 3.abstract 4.structure of the paper 5.examples/use cases 6.research question/formal problem definition
- - - - - - -
% DSR muss nicht sein, kann auch SCRUM oder {"a}hnliches
II. Design Science Research
- DSR method acc. to Peffers and Hevner

\end{comment}
\begin{comment}
2) Ziel der Arbeit, scope of work
- not scope to practically integrate any of the concepts into existing DApp or any other existing system
- welche use cases gibt es f{"u}r ZKPs in RAPADO
- wie k{"o}nnte man diese Umsetzen

4) Ergebnisse skizzieren
- implementation can be a proof of concept, software artifact depends highly on complexity of the use case
\end{comment}