\chapter{Methodology}
\section{Analysis}
SLR
Analysis of other resources
--> ends with requirements
\section{Design}
\section{Implementation}
\section{Evaluation}
(how good/bad the solution is)
-quantitative and qualitative evaluation, e.g. Laufzeit und Interviews

The research concept of the thesis is divided as follows: First, a systematic literature review (SLR) will be conducted according to \cite{HevnerAR2004DSiI, vomBrockeJan2019TDgs, Webster2002AnalyzingTP}. The focus of the SLR are zero-knowledge proof protocols with the scope definition of classification, opportunities, challenges, evaluation methods, and examples in practice. The final search string will be derived from a concept map and the literature found will be summarized through a concept matrix. Knowledge from the first part will be applied to the project RAPADO. The second part of this master's thesis follows a design-oriented approach according to \citet{HevnerAR2004DSiI, PeffersKen2007ADSR}. RAPADO use cases for zero-knowledge proof protocols are investigated, conceptualized, and evaluated, considering aspects found in previously examined literature. The implementation is carried out iteratively in an agile manner, whereby representatives of the aviation industry are consulted.

The initial and current status and results of project RAPADO are reflected on. From this analysis, potential use cases and design requirements for the application of ZKP are derived. The first result following this research concept is a synthesis of literature on the topic of zero knowledge proof protocols. Different ZKP are presented and evaluated. Opportunities and challenges are displayed by considering practical examples. The second part of the research concept results in the implementation of previously examined use cases for ZKP in the RAPADO project in the form of artifacts.

This master's thesis is associated with preliminary work carried out within RAPADO at the department of information systems. Immediate previous research concludes with conceptual solutions and a DApp for aviation industry MRO documentation, centering storage, and traceability \citep{ZedelJ, semesterproject}. Hereby, use cases of uploading, storing, and trading aircraft spare parts certificates were given and specific blockchain platform architectures were used. This research extends previous findings. However, it takes a new perspective by further investigating possible use cases for ZKP as methods of automating verification processes and preserving data confidentiality. The results are expected to represent the entire research project, i.e., beyond previously developed software.

\begin{comment}
I. Systematic literature review according to vom Brocke, Cooper and Webster: 
    1. Definition of Scope
        - classification, examples, challenges and evaluation methods of zero knowledge proof protocols 
        
    2. Conceptualization
        - work with concept map
        - derive at a search string
        
    3. Literature Search and Selection
        - look for review paper first to get good overview about the topic
        a) exclude paper that are too old and have too few citations and/or low impact factor (e.g. 5.5 is high)
        b) exclusion acc. to title and keywords
        c) exclusion acc. to abstract & structure of paper & RQ
        c) exclusion acc. to full text & availability of resource
        
    4. Synthesizing of Literature
        - cluster definitions, examples, drawbacks and evaluation methods (first suggestion can be found in the preliminary agenda)
        - write overview section about ZKP (Chapter 4)
- - - - - - -
How to know if a paper is useful for me?
1.title 2.keywords 3.abstract 4.structure of the paper 5.examples/use cases 6.research question/formal problem definition
- - - - - - -
% DSR muss nicht sein, kann auch SCRUM oder {"a}hnliches
II. Design Science Research
- DSR method acc. to Peffers and Hevner

\end{comment}
\begin{comment}
2) Ziel der Arbeit, scope of work
- not scope to practically integrate any of the concepts into existing DApp or any other existing system
- welche use cases gibt es f{"u}r ZKPs in RAPADO
- wie k{"o}nnte man diese Umsetzen

4) Ergebnisse skizzieren
- implementation can be a proof of concept, software artifact depends highly on complexity of the use case
\end{comment}