\chapter{Conclusion}
In this master thesis, a systematic literature review on zero-knowledge proofs was performed. ZKPs consist of proof systems, which are extended with cryptographic tools and transformations to create argument systems. Adding ZK properties will yield zero-knowledge, succinct, non-interactive arguments of knowledge (zk-SNARKs). Different zk-SNARKs and variation protocols are examined and compared to create a current overview of the most practically used algorithms. The literature review finds that ZKPs are widely applied throughout various fields, grouped into problem-solving domains, while ZKPs have not been discussed with regard to the applicability in the aviation industry. It concludes that current implementation practice evaluates PLONK to be a promising zk-SNARK for decentralized applications. Higher-hanging fruits in ZKP research are obtained through the development of quantum resistant algorithms, and broader adoptions of zk-Rollups and recursive zk-SNARKs.

The RQ is further approached by establishing requirements from previous research within the aircraft maintenance, overhaul, and repair industry. The first requirement of extending ZKP expertise within the RAPADO project is partially satisfied through the systematic literature review. The zk-SNARKs proof and verification of a polynomial example satisfies the first requirement by providing further practical knowledge and application. The second requirement refers to the necessity to not compromise transparency for the need to keep certain competitive MRO data confidential. Although blockchain enables high transparency, the extensive use of ZKP achieves this compatibility of both. The zk-DApp for MRO data attestation and verification is implemented as a software artifact to demonstrate the functionality of PLONK in practice, suggesting a trusted process to securely verify aviation spare part documentation. As an evaluation result, it shows that further implications need to be investigated to fully enable authenticity checks during back-to-birth traceability of MRO documentation. To meet this third requirement of fraud-preventive document authenticity checks, a zero-knowledge data structure is proposed as third artifact. This architecture is based on Merkle trees and ZKPs, and serves as proposal to create secure and appropriate data formats for MRO documentation.
\begin{comment}
hier zusammenfassung klassisch
1 SLR
- examined requirements: ...
2 example calc--> knowledge acc
3 zk-DApp--> transparency confidentiality
4 zk data structure --> appropriate data structure and authenticity 
\end{comment}





\section{Discussion of Results}
%was wird erreicht und dann limitations
This master thesis has implications for scholars and practitioners.
\begin{comment}
    - what was achieved
    -what implications for
    1 zkp field of research
    2 mro industry
\end{comment}

This research is bound to several limitations.
\begin{comment}
    
\end{comment}
\section{Future Work}
%alle punkte sorgfältig ausarbeiten und dann erst losschreiben
\begin{comment}
- zusammenführen der artefakte in RAPADO
- recursive snarks -->auf den ersten Teil der EInleitung beziehen mit so vielen dokumenten zum verifizieren gleichzeitig usw.
\end{comment}