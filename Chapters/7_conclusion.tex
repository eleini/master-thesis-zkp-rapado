\chapter{Conclusion}
This research has implications for scholars and practitioners. 

\citet{efthymiou} provides an outlook for blockchain developers to create more familiarity with the technology, examine and implement widely shared use vases for \acrshort{mro} events, and facilitate standardization and data interoperability. This master thesis provides scholars and practitioners with a topical overview and practical implementations for the aviation industry.

At first, a systematic literature review on zero-knowledge proofs was performed. \acrshort{zkp}s consist of proof systems, which are extended with cryptographic tools and transformations to create argument systems. Adding ZK properties will yield zero-knowledge, succinct, non-interactive arguments of knowledge (\acrshort{zksnark}s). Different \acrshort{zksnark}s and variation protocols are examined and compared to create a current overview of the most practically used algorithms. The literature review finds that \acrshort{zkp}s are widely applied throughout various fields, grouping them into problem-solving domains, while \acrshort{zkp}s have not been discussed with regard to their applicability in the aviation industry. It concludes that current implementation practice evaluates \acrshort{plonk} as a promising \acrshort{zksnark} for decentralized applications. Higher-hanging fruits in \acrshort{zkp} research are obtained through developing quantum-resistant algorithms and broader adoptions of zk-Rollups and recursive \acrshort{zksnark}s.

The research question is further approached by establishing requirements from previous work within the aircraft maintenance, overhaul, and repair industry. The systematic literature review partially satisfies the first requirement of extending \acrshort{zkp} expertise within the RAPADO project. The \acrshort{zksnark}s proof and verification of a polynomial example satisfies the first requirement by providing further practical knowledge and application. 

The second requirement is maintaining transparency while keeping specific competitive \acrshort{mro} data confidential. Although blockchain enables high transparency, further use of \acrshort{zkp}s achieves the compatibility of both. The \acrshort{zkdapp} for \acrshort{mro} data attestation and verification is implemented as a software artifact to demonstrate the functionality of \acrshort{plonk} in practice, suggesting a trusted process to verify aviation spare part documentation securely. 

Further implications must be investigated to facilitate authenticity checks during back-to-birth traceability of \acrshort{mro} documentation. A zero-knowledge data structure is proposed as a third artifact to meet the requirement of fraud-preventive document authenticity checks. This architecture is based on Merkle trees and \acrshort{zkp}s and proposes to create secure and appropriate data formats for \acrshort{mro} documentation.

\section{Discussion of Results}
This research is bound to several limitations. 

The topic of zero-knowledge proofs in blockchain is complex, bringing along resistance to change inside and outside the aviation industry. It is easier to identify appropriate use cases when the technology is well understood.

Specific prototypical software artifacts, e.g., decentralized applications,  are implemented on test networks. It does not require any specific assumptions about the type of network the application will be running on. Still, how different network types affect the developed prototype's functionality must be investigated.

The use case for the \acrshort{zkdapp} needs further maturation to test whether the values and constraints are practical. The demonstration of technical functionalities was at focus. The communication channels between attesters and submitters are shared via browser links in the prototype for demonstration purposes. This approach needs to be more practical and secure, thus, needs to be changed. The mechanism to notify attesters of newly entered values for attestations must be integrated into a broader communication platform with a more secure exchange of notifications for newly created values for attestation.

The zero-knowledge data structure architecture is designed and equipped with roles and process flows but still needs practical implementation and evaluation through tests among industry participants, security modeling, and analysis. The architecture needs to be prototypically incorporated into the \acrshort{zkdapp} and tested.
 
\begin{comment}
-noch die rapado project workshop slides durchschauen
\end{comment}

\section{Outlook}
Future work concluded from this thesis is discussed and sorted from lowest to highest-hanging fruits.

The \acrshort{zkdapp} uses the \acrshort{plonk} implementation in circom and snarkjs. Once the TURBOPLONK implementation is available, the application needs to be updated to use the efficiency increase compared to \acrshort{plonk}. All software artifacts developed within the project RAPADO must be integrated, i.e., the \acrshort{dapp} for documentation storage and traceability and the \acrshort{zkdapp} for trustless documentation attestation and verification. 

The zero-knowledge data structure architecture is a promising base layer for preliminary developed blockchain-based applications in the industry. It must be applied for future prototyping to create a consistent data structure for digitizing aviation spare part data. Furthermore, zero-knowledge proofs must be the focus of technology acceptance modeling in the aviation and \acrshort{mro} industry. Even though it is a technical topic, economic and social analyses concerning change management in the industry need to be performed. 

Recursive \acrshort{zksnark}s become more relevant for this research, considering that millions of documentation events must be captured and verified, i.e., many proofs must be created and verified at once. New research outcomes in this field must be observed, analyzed, and applied to existing implementations once possible.