\thispagestyle{empty}


\vspace*{1cm}

\begin{center}
    \textbf{Abstract}
\end{center}

\vspace*{1cm}
\noindent

The modern aviation industry, with its millions of aircraft parts and numerous stakeholders, faces the challenges of documenting and verifying the maintenance, repair, and overhaul (MRO) events throughout an aircraft's lifecycle. The current paper-based documentation system lacks digital process automation, resulting in disconnected and scattered MRO event entries across various enterprise resource planning systems, compromising transparency and traceability. This situation impedes efficient secondary parts trading and poses challenges in verifying the documentation for airworthiness and traceability in the event of accidents or incidents. To address these challenges, the research project RAPADO aims to drive digitalization in the aviation industry by establishing an industry standard for seamless documentation and verification of aircraft spare parts. This research investigates the applicability of blockchain technology, precisely zero-knowledge proofs (ZKPs), to meet the verification and information confidentiality requirements for aircraft spare parts documentation. Through a systematic literature review, the current state of research on ZKPs is surveyed, providing a comprehensive overview of the most practically used algorithms and their applications in various domains. However, their potential application in the aviation industry still needs to be explored. This master thesis bridges this gap by presenting practical implementations of ZKPs, starting with a step-by-step overview of the Groth16 algorithm through a simple polynomial example. Subsequently, a zero-knowledge decentralized application (zk-DApp) is proposed and evaluated for MRO data attestation and verification to balance transparency and data confidentiality. A fraud-preventive zero-knowledge data structure based on Merkle trees and ZKPs is also introduced to facilitate authenticity checks of spare part certificates. The findings of this research have implications for both scholars and practitioners. The systematic literature review provides an overview of ZKPs. It highlights their potential applications across various fields, while the practical implementations offer concrete solutions for the aviation industry. The proposed zk-DApp and zero-knowledge data structure contribute to the secure and efficient verification of aircraft spare parts documentation, promoting transparency, traceability, and data integrity. Future research directions include broader adoptions of zk-rollups and recursive zk-SNARKs and re-assessing quantum-resistant algorithms' development state and applicability.