\documentclass[12pt,
bibtotoc,liststotoc,appendixprefix
twoside,paper=a4,headings=small]{scrbook}
%
% Packages
% -----------------------------------
\usepackage[
  paper=a4paper,
  left=12.5mm,
  right=25mm,
  top=25mm,
  bottom=50mm,
  bindingoffset=10mm]{geometry}		% Seitenränder und Bindungskorrektur einstellen

\usepackage{amsmath,amssymb,amsfonts}
\usepackage{algorithmic}            
\usepackage{apacite} 				% Literatur-Referenzen: American Psycholog. Assoc.
\usepackage{natbib}	
\usepackage[style=iso]{datetime2}
\setcitestyle{round,aysep={}} 		% Indizierg. in runden Klammern, zw. Autor u. Jahr
\usepackage[latin1]{inputenc} 		% Umlaute im Text
\usepackage[english]{babel}			% Rechtschreibg.
\usepackage[T1]{fontenc}
\usepackage{lmodern}				% Schriftfamilie
\usepackage{microtype}				% für die Mikrotypografie (besseres Schriftbild)
\usepackage{comment}
\usepackage{graphicx} 				% Grafiken einfügen (pdf,png - aber jpg vermeiden)
\graphicspath{{./Pictures/}}          % Pfad zu den Pictures

\usepackage{url}					% URL's formatieren (z.B. in Literatur) 
\usepackage[colorlinks,linkcolor=black,citecolor=black,urlcolor=black]{hyperref} 				% für Hyperlinks in PDF-Dokumenten   
  
\usepackage{tabularx} 				% bessere Gestaltung von Tabellen
\usepackage{longtable} 		
\usepackage{multicol}				
\usepackage{multirow}
\usepackage{booktabs}
\usepackage{tabularx}
		
\usepackage[active]{srcltx}

\usepackage{listings}				% Algorithmen

\usepackage{mdwlist}				% Listen

\usepackage{setspace} 				% Zeileneinstellung
\newtheorem{mydef}{Merksatz}  		% Falls Beispiele, Merksätze m. fortl. Nr. gebr. werden
\newtheorem{bsp}{Beispiel}

\usepackage{todonotes}				% zum Erstellen von ToDos im Editor

\usepackage{lscape}					% zum Rotieren von Seiten

\usepackage{amsmath,amssymb,amsfonts}				% zum Schreiben von mathematischen Formeln

\usepackage{calc}

\usepackage{footnote}				% Fußnoten
\usepackage{tablefootnote}			% Fußnoten in Tabellen

%\clubpenalty = 10000
%\widowpenalty = 10000 \displaywidowpenalty = 10000

\hyphenation{voll-st\"andigen}		% Worttrennungen global definieren

\setcounter{tocdepth}{1}			% Ebenen, die im Inhaltsverzeichnis angezeigt werden

% Document
% -----------------------------------
\begin{document}

\frontmatter 
    % Titelseite soll keine Kopf oder Fußzeile haben
\thispagestyle{empty}

% Alle Elemente sollen zentriert sein
\begin{center}

\vspace*{-10mm}

{\LARGE DEPARTMENT OF \\INFORMATION SYSTEMS\\[1mm]}
FREIE UNIVERSIT\"AT BERLIN\\

\vspace*{1cm}

\includegraphics[width=0.18\textwidth]{fu_logo}

\vspace*{1cm}

% Art der Arbeit
{\Large \textbf{Master Thesis}}\\ 
\vspace{1cm}
{\Large \textbf{Zero-Knowledge Proof Algorithms: A Systematic Literature Review and Application in the Aviation Industry}}\\ 


\begin{comment}
% Titel der Arbeit 
{\Large \textbf{Hier folgt der Titel}}\\ 
\vspace*{1mm}
{\Large \textbf{dieser kann auf bis zu drei}}\\ 
\vspace*{1mm}
{\Large \textbf{Zeilen verteilt werden wenn n\"otig}}\\
\end{comment}
\vspace{1.5cm}

% Name der Autorin
{\LARGE Elina Beletski}\\[15mm]

% Gutachter, Kontaktdaten und Abgabetermin
\parbox{120mm}
{
\begin{large}
\begin{tabbing}
Supervisor: \hspace{.7cm} \=Univ.-Prof. Dr. rer. pol. Natalia Kliewer\\[4mm]
Semester:\> Winter term 2022/23\\
Author:\> Elina Beletski\\ % alphabetische Reihenfolge (Nachname)
Student ID:\> 5504054\\
Address:\> N\"urnberger Str. 18, 10789 Berlin\\
Email:\> elina.beletski@fu-berlin.de\\
Phone:\>+4917620310530\\
Studies:\>M.Sc. Information Systems\\[8mm]
\textbf{Date:} \> \textbf{12 June 2023}\\
\end{tabbing}
\end{large}
}
\end{center}
\clearpage{\pagestyle{empty}\cleardoublepage}
 			% Titelblatt
    \newpage
    \clearpage{\pagestyle{empty}\cleardoublepage}
    \thispagestyle{empty}


\vspace*{1cm}

\begin{center}
    \textbf{Abstract}
\end{center}

\vspace*{1cm}
\noindent

The modern aviation industry, with its millions of aircraft parts and numerous stakeholders, faces the challenges of documenting and verifying the maintenance, repair, and overhaul (\acrshort{mro}) events throughout an aircraft's lifecycle. The current paper-based documentation system lacks digital process automation, resulting in disconnected and scattered \acrshort{mro} event entries across various enterprise resource planning systems, compromising transparency and traceability. This situation impedes efficient secondary parts trading and poses challenges in verifying the documentation for airworthiness and traceability in the event of accidents or incidents. To address these challenges, the research project RAPADO aims to drive digitalization in the aviation industry by establishing an industry standard for seamless documentation and verification of aircraft spare parts. This research investigates the applicability of blockchain technology, precisely zero-knowledge proofs (\acrshort{zkp}s), to meet the verification and information confidentiality requirements for aircraft spare parts documentation. Through a systematic literature review, the current state of research on \acrshort{zkp}s is surveyed, providing a comprehensive overview of the most practically used algorithms and their applications in various domains. However, their potential application in the aviation industry still needs to be explored. This master thesis bridges this gap by presenting practical implementations of \acrshort{zkp}s, starting with a step-by-step overview of the Groth16 algorithm through a simple polynomial example. Subsequently, a zero-knowledge decentralized application (\acrshort{zkdapp}) is proposed and evaluated for \acrshort{mro} data attestation and verification to balance transparency and data confidentiality. A fraud-preventive zero-knowledge data structure based on Merkle trees and \acrshort{zkp}s is also introduced to facilitate authenticity checks of spare part certificates. The findings of this research have implications for both scholars and practitioners. The systematic literature review provides an overview of \acrshort{zkp}s. It highlights their potential applications across various fields, while the practical implementations offer concrete solutions for the aviation industry. The proposed \acrshort{zkdapp} and zero-knowledge data structure contribute to the secure and efficient verification of aircraft spare parts documentation, promoting transparency, traceability, and data integrity. Future research directions include broader adoptions of zk-rollups and recursive \acrshort{zkp}s and re-assessing quantum-resistant algorithms' development state and applicability.
    \newpage
    \clearpage{\pagestyle{empty}\cleardoublepage}
    \onehalfspacing                  	% Zeilenabstand ab hier 1.5 zeilig
    \tableofcontents 					% Inhaltsverzeichnis
    \clearpage{\pagestyle{empty}\cleardoublepage} 
    
    \listoffigures 					 	% Abbildungsverzeichnis
    \clearpage{\pagestyle{empty}\cleardoublepage}
    
    \listoftables						% Tabellenverzeichnis rein
    \clearpage{\pagestyle{empty}\cleardoublepage}
% -----------------------------------
\mainmatter 							% die einzelnen Chapters
    \chapter{Introduction}
\citet{kliewer2005optimierung}
\citet{semesterproject}
\section{Motivation}
A modern civil aircraft consists of approximately three million parts, of which thousands of components must be maintained, repaired, and overhauled (MRO) throughout its life-cycle. These MRO events have to be documented. If aircraft and accompanying components demonstrate safe technical conditions, aviation authorities declare them airworthy; hence, they allow them to operate \citep{FornaconFrank}. Considering there are 25,000 commercial aircraft in operation and approximately 20,000 suppliers in the industry, there are hundreds of millions of events to be documented and verified \citep{mroBCservices1}.

The aircraft part life-cycle starts with initial manufacturing documentation and continues with buying, selling, leasing, repair and overhaul documentation, as well as disposal documentation \citep{FornaconFrank, mroBCservices}. To this day, back-to-birth documentation of aircraft and components is still largely paper-based and lacks digital process automation \citep{efthymiou}. Transparency and traceability are compromised through disconnected, scattered MRO event documentation entries across various enterprise resource planning systems operated by the industry's participants \citep{FornaconFrank, mroBCservices1}. Aircraft components without verified back-to-birth documentation history have no value, i.e., more sustainable second-hand trading of parts is not possible. This overall situation has an immediate effect on the verification of aircraft parts documentation by aviation authorities: verifying for airworthiness or traceability in the event of an accident is a cumbersome, inquiry-based process, which leaves room for withholding of information and counterfeit parts \citep{planecrash, efthymiou}.

\textit{RAPADO} is a research project that is driving digitalization in the aviation industry by creating an industry standard for seamless documentation and verification of aircraft spare parts. The research is being conducted by the Information Systems Department at Freie Universit{/"a}t Berlin in collaboration with project partner Opremic Solutions GmbH and is being funded by the German Federal Ministry of Economic Affairs and Climate Action. Preliminary work at the department resulted in further research in the application of blockchain technology and encryption methods with the aim of targeting the requirements of documentation storage and traceability. Previously, a DApp was developed, where aircraft part certificates can be stored in a distributed file storage (IPFS) and traced via the file hash, which is persisted on an Ethereum blockchain.

\subsection{Outline of Contribution}
The aviation industry is characterized by high competition and mistrust among the stakeholders \citep{Chatzi2019TDoC}, which creates conflicting expectations when pursuing the goal to digitalize and automate its processes. Aircraft parts documentation history should always be verifiable and tamper-proof, while preserving data confidentiality and protecting the identity and ownership of industry participants \citep{Wickboldt2019BlockchainFW}. Facilitating such a verification process without revealing sensitive information can be achieved through zero-knowledge proof (ZKP) algorithms.

This master's thesis focuses on the requirements of verification and information confidentiality within project RAPADO. It extends previous findings by reviewing current literature on the topic of zero-knowledge proof protocols and practical examples. Additionally, suitable use cases and their implementation are investigated. The research question is formulated as follows:

\textit{Which requirements and use cases for zero-knowledge proof protocols can be discovered in the RAPADO project and how can they be implemented?} 
%Adam-Air-Flug 574 Beispiel
%formulate topic of master thesis, research question(s)
\begin{comment}


c)research question
d)Relevanz/Contribution: what is my contribution to the topic
-->1)artifact(s)
\end{comment}
\begin{comment}
1.1 Motivation = Problem

a)how do I start a problem motivation? introduce the domain! -to give readers some time to breathe, especially 
when they're coming from a different research area

b)introduce the problem
- if I talk about the problem, I never talk abot the solution
- start with zero, for readers that have almost no knowledge in the field, write it in an understandable way--> get more detailed
to the end
-give examples/use cases (not a must)-->something the readers has heard of, as easy as possible but as complex as possible 
to later reference it and build it up
-why is it important to solve this problem?

c) (Research Question) for us because it's a seminar paper: formal problem definition!
2 properties of RQ and problem defition: exact and complete (what the paper is about)
what's the difference between FPD (formal problem definition) and RQ tho?
\end{comment}

\begin{figure}
	\centering
		\includegraphics[width=0.7\textwidth]{./Pictures/bsp1.png}
	\caption{Beispiel 1 zum Einf{/"u}gen einer Grafik}
	\label{fig:bsp1}
\end{figure}

\section{Outline of Contribution}
\begin{comment}
a)what is the aim of the thesis
b)outline of contribution (what is your contribution to the problem? -->solve a problem, give an overview,
implement something)
in technical fields, there are two parts: 
1) artifact (software, knowledge etc.) 
2) methodological approach/reasoning on the way to the artifact-->FOR US: SYTEMATIC LITERATURE REVIEW (our only methodology)
\end{comment}

\section{Structure}
\begin{comment}
ALWAYS START WITH THIS SENTENCE: "This paper/article/etc. is structured as follows." (kein doppelpunkt)
"In Section 2, related work of XYZ is introduced.."
\end{comment}



    \clearpage{\pagestyle{empty}\cleardoublepage}		% löscht Kopfzeilen und Seitennummerierung von der letzten Seite eines Chapters, sofern dort kein Text mehr steht
    \chapter{Related Work}
\begin{comment}
-knowledge base I build upon
- related papers can be: content related, methodology, technical -->structure accordingly

(in a thesis, it is mandatory to have Related Systems, when you implement an application)
- Sedlmeier, V{/"u}lter, Str{/"u}ker (2021): trading green energy certificates: same problem of verification vs privacy, Implementation architecture based on zk RollUps, very good reasoning about requirements (same as Rapado), maybe a good starting point to create an architecture as second artifact for rapado? architecture to verify information about certificates
\end{comment}
Preliminary research at the Department of Information Systems focuses on the architecture of suitable blockchain-based platforms for aviation industry MRO documentation. \citet{WickboldtMeiseKliewer} proposes a framework to use a private blockchain-based architecture in HyperLedger Fabric, whereby node registration is managed through trusted Membership Service Providers (MSP). This first approach resulted from initial core requirements of data persistence, selective data access, data integrity, and transparency of back-to-birth documentation histories \citep{WickboldtClemens2018BzdD}. Subsequently, these core requirements were further specified through extensive information exchange with previously identified stakeholder groups of airline companies, MRO full-service providers, and MRO parts merchants \citep{ZedelJ}. Further research proposed a public blockchain-based platform with smart contracts and a decentralized file storage system \citep{semesterproject, ZedelJ}. MRO documentation is stored in the decentralized file storage system, which produces a unique file hash. It also identifies the document that is implemented as a non-fungible token (NFT) in a smart contract. Following the identified process, the validation status set by aviation authorities is captured. In a DApp software artifact, aviation authorities get full document access using threshold encryption. They receive their key shares from the smart contract and can decrypt them using their private key. By combining their key shares, the corresponding document can be decrypted, accessed, and used for verification \citep{semesterproject}. Hence, access management is not central because various nodes in the network have to combine their key shares. However, once access is granted, all data is exposed. Documents contain competitive information and must be treated confidentially. This led to further research on how to verify MRO documentation without fully exposing sensitive information. Zero-knowledge proofs were identified as promising technology to meet the requirement of data confidentiality during the verification of MRO documents on a public blockchain-based platform \citep{ZedelJ}. This master's thesis extends previous research by focusing on the topic of ZKP. Use cases are examined and implemented with regards to business processes and requirements developed within project RAPADO.
%content related, methodology, technical -->structure accordingly
%on chain computations are expensive (see project semester), ZKPs are powerful off chain solutions for verification processes

%look at some review paper in WOS and tease them
%present previous work (Wickboldt, Zedel, Kliewer, semester project)
    \chapter{Methodology}
\section{Analysis}
SLR
Analysis of other resources
--> ends with requirements
\section{Design}
\section{Implementation}
\section{Evaluation}
(how good/bad the solution is)
-quantitative and qualitative evaluation, e.g. Laufzeit und Interviews

The research concept of the thesis is divided as follows: First, a systematic literature review (SLR) will be conducted according to \cite{HevnerAR2004DSiI, vomBrockeJan2019TDgs, Webster2002AnalyzingTP}. The focus of the SLR are zero-knowledge proof protocols with the scope definition of classification, opportunities, challenges, evaluation methods, and examples in practice. The final search string will be derived from a concept map and the literature found will be summarized through a concept matrix. Knowledge from the first part will be applied to the project RAPADO. The second part of this master's thesis follows a design-oriented approach according to \citet{HevnerAR2004DSiI, PeffersKen2007ADSR}. RAPADO use cases for zero-knowledge proof protocols are investigated, conceptualized, and evaluated, considering aspects found in previously examined literature. The implementation is carried out iteratively in an agile manner, whereby representatives of the aviation industry are consulted.

The initial and current status and results of project RAPADO are reflected on. From this analysis, potential use cases and design requirements for the application of ZKP are derived. The first result following this research concept is a synthesis of literature on the topic of zero knowledge proof protocols. Different ZKP are presented and evaluated. Opportunities and challenges are displayed by considering practical examples. The second part of the research concept results in the implementation of previously examined use cases for ZKP in the RAPADO project in the form of artifacts.

This master's thesis is associated with preliminary work carried out within RAPADO at the department of information systems. Immediate previous research concludes with conceptual solutions and a DApp for aviation industry MRO documentation, centering storage, and traceability \citep{ZedelJ, semesterproject}. Hereby, use cases of uploading, storing, and trading aircraft spare parts certificates were given and specific blockchain platform architectures were used. This research extends previous findings. However, it takes a new perspective by further investigating possible use cases for ZKP as methods of automating verification processes and preserving data confidentiality. The results are expected to represent the entire research project, i.e., beyond previously developed software.

\begin{comment}
I. Systematic literature review according to vom Brocke, Cooper and Webster: 
    1. Definition of Scope
        - classification, examples, challenges and evaluation methods of zero knowledge proof protocols 
        
    2. Conceptualization
        - work with concept map
        - derive at a search string
        
    3. Literature Search and Selection
        - look for review paper first to get good overview about the topic
        a) exclude paper that are too old and have too few citations and/or low impact factor (e.g. 5.5 is high)
        b) exclusion acc. to title and keywords
        c) exclusion acc. to abstract & structure of paper & RQ
        c) exclusion acc. to full text & availability of resource
        
    4. Synthesizing of Literature
        - cluster definitions, examples, drawbacks and evaluation methods (first suggestion can be found in the preliminary agenda)
        - write overview section about ZKP (Chapter 4)
- - - - - - -
How to know if a paper is useful for me?
1.title 2.keywords 3.abstract 4.structure of the paper 5.examples/use cases 6.research question/formal problem definition
- - - - - - -
% DSR muss nicht sein, kann auch SCRUM oder {"a}hnliches
II. Design Science Research
- DSR method acc. to Peffers and Hevner

\end{comment}
\begin{comment}
2) Ziel der Arbeit, scope of work
- not scope to practically integrate any of the concepts into existing DApp or any other existing system
- welche use cases gibt es f{"u}r ZKPs in RAPADO
- wie k{"o}nnte man diese Umsetzen

4) Ergebnisse skizzieren
- implementation can be a proof of concept, software artifact depends highly on complexity of the use case
\end{comment}
    \chapter{The RAPADO Project}
\section{Project Introduction}
Project RAPADO is part of the aviation research program in the German Federal Ministry of Economic Affairs and Climate Action, which, inter alia, supports research and development of disruptive technologies to be implemented in the aviation industry within the next 20 to 30 years. Main goal of this project is conducting research on seamless, complete, and safe documentation and certification of aircraft spare parts materials using current developments in blockchain technology. The common practice to reuse, repair and trade aircraft spare parts is strengthened, as well as safely and permanently transformed. The MRO industry operates in an error-prone and non-digital manner: If a spare part is repaired and ready to be returned, the corresponding certificates and receipts are sent via mail, i.e., purchasing complex spare parts results in paper-based documentation being delivered on pallets. Hence, MRO providers are not motivated to reuse spare parts and rather buy new to prevent risks and liabilities due to incomplete documentation, because spare parts are not of any use for civil transportation without complete documentation history. The project consortium, consisting of representatives from Freie Universit{\"a}t Berlin and industry partner Opremic Solutions GmbH, aims at creating a digital process to reduce production of new spare parts, enable trade, and provide an increased security through persistent documentation, while contributing to the funding target of productive and efficient aerospace. The industry standard is distributed as open-source, driven by critical mass of users and the International Air Transportation Association (IATA) as partner.

\section{Research Status}
In consideration of preliminary work of \citet{ZedelJ, Wickboldt2019BlockchainFW, semesterproject} in the project, the project status is reflected on and current project requirements are stated as basis for this thesis' outcomes.

Recent project outcomes suggest that the adoption of DLT-based solutions for MRO documentation is dependent on full compliance of requirements set by the operating model in the aviation industry, whereby the main compatibility aspects are combining the benefits of data persistence and privacy while capturing spare part documentation. Private decentralized networks were suggested \citep{Wickboldt2019BlockchainFW}. Data persistence describes the constant accessibility of spare part documentation and MRO event history. In an industry context, it is often referred to as tamper-proof data storage and retention. Decentralized networks increase data availability, which needs further assessment with regards to DLT-based solutions in the project. High availability is seen as a result of a high number of nodes operated in the network, which also bears the risk of redundancy and efficiency loss. Further research of off-chain possibilities can help counteract this risk.

-look further to semester project intros: 
(i) High speed of the system to not intervene with processes of part usage.
(ii) Consolidation of parts and certificates from different sources.
(iii) Persistence of data while operating in a trust-free and permission-based environment to secure anti-counterfeiting and confidentiality.

next:
-security and privacy
-transparency
- concepts for privacy identified: ZKP as one outcome of previous research for future research
- for this, three main requirements resulted: 1)knowledge accumulation 2)privacy vs transparency 3)digitization and data formats fit for purpose
--> use backings from project literature used also in semester project and jans publication
\begin{comment}
-go thru zedel kliewer and list principles, say that zkps were one solution criteria mentioned already
\section{Project Introduction}
\subsection{Background}
\subsection{Initial Problem Analysis}
\subsection{Previous Work}
\subsection{Reflection}
\section{Current Project Status}
\subsection{Problem Analysis}
\subsection{Design Requirements}
\subsection{Derived Use Cases}

\begin{table}[bp]
	\centering
		\caption{Beispiel 1 zum Einfügen einer Tabelle}
		\begin{tabular}{| c c c |}
		\hline
			&&\\
			Monat & Linie & Minuten\\
			\hline
			\hline
			&&\\
			Jan & U7 & 10 \\
			Feb & U9 & 12 \\
			Mär & U9 & 20 \\
			\hline
		\end{tabular}

	\label{tab:Beispiel1}
\end{table}


\begin{figure}[tp]
	\centering
		\includegraphics[width=0.50\textwidth]{./Bilder/bsp2.png}
	\caption{Beispiel 2 zum Einfügen einer Grafik}
	\label{fig:bsp2}
\end{figure}
\end{comment}

    \chapter{Zero-Knowledge Proofs}
This chapter surveys selective literature about zero knowledge proofs for practical design application. The goal is to familiarize the reader with the content classification of zero-knowledge proofs in cryptography, and to give an introduction and comparison of main argument systems widely used as of today. Zero-knowledge proof systems belong to the domain of verifiable computation \citep{Simunic}. Verifiable computation (VC) makes use of cryptographic protocols and arguments to verify that a computation was performed correctly. Arguments allow for false proofs only if they require very high computational power. 

The introduction of interactive proof systems (IPs) by \citet{GoldwasserIPs} and \citet{BabaiIPs} shows that only correct proofs are valid and malicious proving strategies cannot be verified. Traditional proofs are static and are made for easy, step-wise computation verification, whereas IPs require interaction between prover and verifier. In computational complexity theory, interactive proof systems are abstract machines exchanging messages between prover and verifier to convince the verifier that these sets of strings belong to a language \(L\). Here, the formal language \(L\) is defining a decision problem, i.e., a computational problem that has only two outputs, yes and no. Examples of sets of strings in \(L\) can be, that a certain Bitcoin transaction is valid, \(x = 7\) for a given \(f(x)\), or a certain object belongs to a specific merkle tree. The untrusted prover has unlimited computational power and the verifier is honest and computationally restricted. Advances in computational complexity theory during that time showed that IPs are more efficient and belong to a wider class than the traditional \textbf{NP}, i.e., problems solvable in deterministic polynomial time by reading proof strings of polynomial length \citep{SassonIOPsinproceedings}. In 1987, it was shown that every language belonging to NP has zero-knowledge proof systems \citep{anymental10.1145/28395.28420}. Later, it was shown that the class of \textbf{IP}, i.e., problems solvable by interactive proof systems, lies in \textbf{PSPACE}, i.e., problems solvable in polynomial space \citep{Shamir10.1145/146585.146609}, and that every language \(L\) in polynomial time has an interactive proof system \citep{Lund10.1145/146585.146605}. The complexity class of IP describes prover and verifier interaction in a polynomial number of rounds. Other important advances in computational complexity theory are \textbf{MIP=NEXP} \citep{Laszlo} and the \textbf{PCP theorem} \citep{PCPTheorem}. These works resulted in a set of proof and argument systems, that will be introduced in this chapter. \citet{Thaler} gives a further, exhaustive overview of all systems and in-depth protocol descriptions. 

The examined proof systems are secure against computationally unrestricted provers: \textit{Interactive Proof (IP), Multi-prover Interactive Proof (MIP), linear Probabilistically Checkable Proof (PCP), and Interactive Oracle Proof (IOP)}. 

Combining the systems above with cryptographic tools to force certain behavior in the proof generation will create an argument system. Argument systems are considered to be zero-knowledge, if and only if the proof does not reveal anything but its own validity \citep{GoldwasserIPs}. Adding certain properties, e.g. non-interactivity, succinctness, and knowledge, will design a certain argument of knowledge, e.g. zk-SNARK (zero-knowledge succinct non-interactive argument of knowledge). Different zk-SNARKS and notions will be examined in more detail.

The next sub chapters are structured according to the design approach described above. First, IPs, MIPs, PCPs, and IOPs will be defined. Through the combination with polynomial commitment schemes argument systems can be designed. Non-interactivity is achieved through Fiat-Shamir transformation. Second, properties and mathematical tools will be introduced to describe different arguments of knowledge: zk-SNARK, FRI-STARK and bulletproofs. Third, real-world applications of zero-knowledge proof systems will be summarized. Lastly, the different proof systems will be evaluated in different application scenarios.

\section{From Proof Systems towards Argument Systems}

A mathematical proof in the context of computer science and cryptography is any object that convinces a verifier that a statement is correct. Mathematical proofs embody what is defined in the complexity class NP. A proof system is a structured scheme that outputs a decision of whether a statement is correct or incorrect. In general, there are three properties that are desirable for proof systems \citep{GoldwasserIPs}, which will be introduced shortly. Throughout this chapter, these properties will be revisited more exhaustively.

\begin{itemize}
    \text{\parbox{420pt}{
    \item The procedure to create and verify proofs should be \textit{efficient and fast}. 
    \item \textit{Completeness}: True statements should have a convincing proof of their validity.
    \item \textit{Soundness}: If a statement is incorrect, there is no possibility of it having a convincing proof.}}
\end{itemize}

Unlike argument systems, proof systems do not limit the malicious prover in its computational power (\textit{statistical soundness}). Using cryptographic primitives and restricting the prover, e.g., probabilistic polynomial time proof, so that it cannot break the primitives, describes the design towards argument systems, which are computationally sound \citep{ArgSystems, MicaliArgSys}. Each proof system presented makes assumptions on the prover. However, only with the use of cryptographic tools and zero-knowledge, these proof systems can be extended with additional properties to yield zero-knowledge argument systems of various kinds.

\begin{comment}
IPs are the basis to understand cryptographic protocols and argument systems
argument systems are IP with a specific assumption
polynomial IOPs together with polynomial commitment scheme always yields a specific argument of knowledge, can be made non interactive with fiat-shamir, can be succinct too, different abwandlungen
but before we need basic understanding on different proof systems that lie behind the construction of argument systems
then we need understanding of commitment scheme and fiat shamir transform
how zero knowledge is applied in the design will be shown later when going through the different systems

information secure = only theoretically
what are those, little history
1:polynomial IOPs 
- interactive proofs intro
- MIPs
- constant round IOPs
2:linear PCPs
-intro from the book Thaler-->combined with cryptographic primitives we get SNARKs, everything is a SNARK and SNARK variation
-Luong and Park: properties intro
- Yang Yang et al properties
-Soonhyeong 2021: better verification with EVM to verify non-maliciousness of blocks (zKSNARKS used)
\end{comment}

\subsection{Interactive Proofs}

In interactive proof systems, the prover with unlimited computational resources interacts with a computationally bound verifier in order to convince the verifier of the correctness of a statement. The verifier randomly challenges the prover, which is referred to as coin tosses \citep{GoldwasserCoinTosses}. These challenges happen in rounds, until there are sufficient tests run and the verifier is convinced. \citet{GoldwasserIPs} defined an interactive proof system that is private, i.e., that the verifiers challenges are not publicly accessible, whereas the interactive proof system of \citet{BabaiIPs} allows the coin tosses to be publicly accessible by the prover.
\begin{align*}
    \text{\parbox{450pt}{\textbf{Interactive Proof System.} \textit{\(L\) is a language over \(\{0,1\}^n\), with n representing the input size domain of n-bit strings. An interactive protocol is a interactive proof system if after \(k\) rounds, the probabilistic verifier in polynomial time exchanged \(k\) messages with the computationally unrestricted prover, and has to either accept or decline the correctness of the prover's proposition. IP is the complexity class of problems solvable by a k-round interactive proof system.}}}
\end{align*}
The transcript is the order in which messages are exchanged. The prover and verifier are functions \(P(x), V(x)\) with common input \(x\). The overall distribution of all transcriptions between prover and verifier is called \(View_V(P(x), V(x))\), which is bound by the number of rounds between \(P\)and \(V\). \(P\) provides a result that satisfies the proposition, e.g., \(y\) to a function \(f(x) = y\). \(P\) and \(V\) exchange a transcript of messages \((m_1, m_2, m_3, ..., m_k)\), whereby both parties take turns and the last message is sent by the prover. Note, that \(V\) is probabilistic with internal randomness \(r\). Hence, the output depends on \((V, x, r, P)\) and is \(\{0,1\}\), i.e., \(1\) if the statement is correct, and \(0\) if it is incorrect \citep{GoldwasserIPs, BabaiIPs}. Interactive proof systems have completeness error \(\delta_c\) and soundness error \(\delta_s\) . For any input \(x\), there must be a convincing proof that \(f(x)\) is correct. Incorrect statements for \(y \neq f(x)\) cannot result in a convincing proof, i.e., a malicious prover does not exist IP systems are valid if \(\delta_c, \delta_s \leq 1/3\) \citep{Thaler}.
\subsubsection{Non-interactive, publicly verifiable arguments}
The \textit{Fiat-Shamir transformation} takes any interactive proof system based on public coin tosses and transforms it into a non-interactive, publicly verifiable system. This transformation can be described effectively with the use of an ideal cryptographic assumption called \textit{Random Oracle Model (ROM)}. 
\subsubsection{Random Oracle Model}
The ROM methodology was introduced in cryptographic theory to satisfy the goal of designing secure cryptographic protocols. First, an ideal system is designed, whereby all parties have access to an oracle, i.e., a public random function. Once the security of the protocol is proven, the random oracle function is replaced by a cryptographic hashing function to implement the system in practice \citep{ROMBellare}. In IPs, the random oracle function is part of the view, i.e., random choices of challenges simulator output. It is assumed that prover and verifier can query the random oracle function \(f_R\) by sending an input \(x\), so that the random oracle returns \(f_R(x)\). In ROM, the efficiency is dependent on x, i.e., for every input in the problem size domain \(D\) , \(f_R(x)\) is captured. Since it is only practical in theory, because \(|D|\) is large to ensure security, e.g., \(2^{256}\), hash functions are used in practice \citep{ROMBellare, Thaler}, e.g., POSEIDON or SHA-3. The ROM is controversially discussed in retrospective: On the one hand, it is argued that there exist protocols that are only secure in the random oracle model and all implementation efforts in practice have led to insecure protocols, whereby the security of the ideal scheme is not maintained. The area of work at focus are digital signatures and public-key encryption, showing that the ROM is not sound \citep{ROMCanetti}. On the other hand, this conclusion is discussed to indicate there is no evidence of practical security weaknesses if there is a need for random oracle systems in a protocol. This statement is supported by protocol design examples, e.g., Elliptic Curve Digital Signature Algorithm, ECDSA, that have led to cryptographic insecurity without the use of a random oracle \citep{ROMretroKoblitz}. 
\subsubsection{Fiat-Shamir Transformation}
The Fiat-Shamir transformation \citep{ROMFiat1986HowTP} removes the need for prover-verifier interaction in a public coin interactive proof system with the help of the random oracle function. The result is a non-interactive, publicly verifiable system. IPs send messages between prover and verifier, whereby the verifier challenges the prover at random. Every message sent by the verifier in the IP is replaced by values obtained by querying the random oracle. The query always depends on the previous message sent by the prover to prevent soundness attacks. In summary, the prover only sends one message containing the transcript, i.e., the list of all messages sent by the prover and the query results of the random oracle. The input x has to be appended to the list every round. The verifier's coin tosses are not needed any longer, and there is no need for the verifier to send messages to the prover (see Figure \ref{fig:FST}).

\begin{figure}[hbt]
	\centering
		\includegraphics[width=0.8\textwidth]{Pictures/FST.png}
	\caption{Generic Fiat Shamir Transformation illustrated (based on \citet{Thaler})}
	\label{fig:FST}
\end{figure}
Argument systems are obtained by applying polynomial commitment schemes to proof systems, whereby the prover commits to a low-degree polynomial. It allows for polynomial evaluation verification without possessing all the information of this polynomial.  Polynomial commitment schemes generically are used to proof that a polynomial, evaluated at a specific input results in a specific output. The prover commits to the polynomial, which can be perceived as some object hiding the polynomial, e.g., similar to a hash. The verifier challenges the commitment with a random value. The committer then creates a proof that the polynomial evaluates at that random value at a specific point. The polynomial itself is not revealed. In interactive proof systems with an honest prover running in polynomial time, e.g., GKR protocol, any arithmetic circuit can be evaluated \citep{GKR10.1145/1374376.1374396}. Arithmetic circuits are gates that operate on two types, i.e., addition and multiplication. Retrospectively, it is an important achievement, since arbitrary computer programs can be expressed via arithmetic circuits, which is the basis layer of many zero-knowledge protocols as of today. More mathematical tools for understanding the functionality of argument systems are covered in chapter 5.2.
\begin{comment}
introduced by  GOLDWASSER, Shafi; MICALI, Silvio; RACKOFF, Charles. The
knowledge complexity of interactive proof systems. SIAM Journal
on computing. 1989, vol. 18, no. 1, pp. 186–208.
GKR protocol-->why? GKR protocol is general-purpose in the sense that it solves the problem of arithmetic circuit evaluation, and any problem in P can
be “efficiently” reduced to circuit evaluation
randomness
completeness
soundness
statistical soundness
computational soundness
fiat shamir transformation for non interactive
knowledge soundness
proof of knowledge
obtaining zero knowledge from it: verifier learns nothing about the witness
\end{comment}

\subsection{MIPs}
\subsection{Linear PCPs}
\textit{Probabilistically Checkable Proofs} (PCPs) differ from previously introduced proofs, because there is no need for the prover to answer queries based on the current or previous query content posted by the verifier. The polynomial time verifier is provided with oracle access to a static proof string \begin{math} \pi \end{math}, whereby the proof is queried and the result of it only depends on the currently processed query \citep{PCP}. The breakthrough of powerful, pairing-based schemes is attributable to \citet{GennaroLinPCP}, who introduced \textit{Quadratic Arithmetic Programs} (QAPs), as variant of quadratic span programs. Quadratic span programs are more efficient, because they only satisfy boolean circuits. Therefore, the introduction of QAPs is important to represent more practical computations, e.g., multiplication gates, whereby the efficiency lies in the usability for effectively solving more natural problems. Any arithmetic circuit instance can be transformed into instances of a \textit{rank-1 constraint system} (R1CS).
\begin{align*}
    \text{\parbox{450pt}{\textbf{R1CS.} \textit{A R1CS is an intermediate representation of the computational problem, which is used to perform the application of argument systems. Given a set of \(n \times m \text{ matrices} \ A, B, C\), with values derived from a finite field \begin{math} \mathbb{F} \end{math}. A R1CS instance is called satisfiable, if there exists a solution vector \(z \in\) \begin{math} \mathbb{F}^n \end{math} with \(z_1 = 1\), so that}}}
\end{align*}
\begin{align*}
    (A \cdot z) \circ (B \cdot z) = C \cdot z.
\end{align*}
For every \(i\)th row each of the three matrices, belonging to the finite field circuit size, the following equation must hold true:
\begin{align}
    \langle a_i, z \rangle \cdot \langle b_i, z \rangle - \langle c_i, z\rangle = 0 
\end{align}

In practice, \(z\) is known by the prover. The solution vector also has a public input, namely the result to a computation (refer to chapter 6.1 for example calculations). The goal is to arrive at an univariate polynomial \(t(x)\), when divided by the minimal polynomial, represents a secret polynomial \(h(x)\) without remainder. The minimal polynomial is always known, if the number of constraints are known, and is a multiple of \(t(x)\). In the linear PCP, (5.1) is evaluated in linear, constant time. The QAP is obtained by taking each value of each row of the R1CS matrices as output of a polynomial, which is to be calculated. The polynomial results to that specific value in the R1CS, when evaluated at \(X = \{1, 2, ..., \text{number of constraints}\}\). Given these constraints, the three sets of polynomials are calculated via the sum of \textit{Lagrange Interpolation}.
\begin{align*}
    \text{\parbox{450pt}{\textbf{Lagrange Interpolation.} \textit{The polynomial obtained through Lagrange Interpolation is the polynomial \(P(x)\) of degree at most \(\leq (n-1)\) and passes through the n points \((x_1, y_1 = f(x_1)), (x_2, y_2 = f(x_2), ..., (x_n, y_n = f(x_n))\), that are given. It is denoted by}}}
\end{align*}
\begin{align}
        P(x) = \sum_{j=1}^{n} P_j(x)
\end{align}
\begin{align*}
        P_j(x) = y_j * \prod_{\substack{k = 1 \\ k \neq j}}^{n}\frac{x-x_k}{x_j-x_k}
\end{align*}
\begin{align*}
    \text{\parbox{450pt}{\textbf{Example.} \textit{Let \(n = 2\) with the three given points \([1,0], [2,1], [3,0]\). The polynomial obtained through Lagrange Interpolation is: 
    }}}
\end{align*}
\begin{align*}
    P(x) = \sum_{j=1}^{2}\left(y_j*\prod_{\substack{k = 1 \\ k \neq j}}^{n}\frac{x-x_k}{x_j-x_k}\right)
    &= 0 * \frac{(x-2)(x-3)}{(1-2)(1-3)}+1*\frac{(x-1)(x-3)}{(2-1)(2-3)}\\
    &= 0 * \frac{(x-1)(x-2)}{(3-1)(3-2)} = \frac{x^2-4x+3}{-1}\\
    &= -x^2+4x-3
\end{align*}
Each gate is the \(x\) value and the values of the R1CS matrix column are the corresponding \(y\) values for the computation, so that the QAP can be obtained. It consists of three sets of polynomials (see chapter 6.1 for example calculations). Each interpolation calculation receives n points (number of constraints), and the result is a polynomial of degree \(n-1\). Multiplying each polynomial in the matrices with the solution vector yields the final polynomials \(A(x), B(x), C(x)\), whereby 
\begin{align}
    \frac{A(x) * B(x) - C_(x)}{Z(x)} = H(x).
\end{align}
In a linear PCP, the proof consists of evaluations of linear functions, which is created by the prover. Soundness is guaranteed against a dishonest prover by ensuring that the proofs are relying on a specific structure. The verifier is querying the proof. By transforming linear PCPs into non-interactive argument systems, the trusted setup issues proving and verifying keys, which are used to perform checks using pairing-based cryptography. Many systems rely on this feature, e.g., Groth16, whose mathematical tools and general protocol procedures are presented in 5.2.1.

\subsection{constant round IOPs}
\subsection{Polynomial IOPs}
%(focus on chapter 10.6, combine with preliminaries from previous chapters in Thaler)

\section{Zero-knowledge non-interactive arguments of knowledge}


\begin{comment}
- what is zero knowledge definition
- what means indistinguishable
- different types of zero-knowledge (perfect, statistical, computational)
-first succinct argument of knowledge by goldwasser etc. 
-what is succinct
- repeat that Fiat Shamir Transformation can make them all non-interactive
-every proof from above combined with polynomial commitment schemes yields a SNARK
-short explanation what are polynomial commitment schemes: %https://coingeek.com/how-plonk-works-part-1/#:~:text=PLONK%20is%20a%20state%2Dof,by%20all%20circuits%5B1%5D.
- how: 1 design a public-coin polynomial IOP for circuit- or R1CS-satisfiability, 2 use polynomial commitment scheme 3 remove interaction via fiat shamir
- linear PCPs are exception: they need to be combined with pairing based cryptography
- everything is a sub of SNARKs
- I only want to cover three most practical zero knowledge SNARKs
- Constant-round IOPs vs. MIPs and IPs-->p.301 Thaler
- every combo with polynomial commitment scheme is a SNARK
- we focus on Groth16, PLONK, FRI-STARK, bulletproof

NITULESCU, Anca. zk-SNARKs: A Gentle Introduction [https:
/ / www . di . ens . fr / ~nitulesc / files / Survey - SNARKs . pdf].
2020. Tech. rep. ENS Paris. Accessed: 2022-05-15.
\end{comment}

\subsection{Circuit-specific zk-SNARK}
\begin{comment}
- implementation of zero-knowledge argument systems was not possible until very recently, with advances in blockchain technology
- Linear PCP and pairing based cryptography

- definitions
- trusted setup, CRS, toxic waste
- FFT and Elliptic Curve Pairing
- main functionalities: trusted setup, proof and verify
- history
- Groth16
- performance enhancements by Groth\& Maller 2017, Lipmaa 2019 and Kim Lee Oh 2020 as the newest contribution with only one single verification
- Luong and Park easy Intro
- Yang Yang et al: Intro Groth16, R1C1, QAP easy examples
- Baghery et al. mathematical backings if needed
- Benamara: ECs and pairings, good math paper 
- Guo et al: QAP, Bilinear Maps, R1CS theory
\end{comment}



\subsection{Towards Universal Setup zk-SNARK}
\textit{Permutations over Lagrange-bases for Oecumenical Non-interactive Arguments of Knowledge} (PLONK) is a zk-SNARK with a universal trusted set-up, which produces a \textit{structured reference string} (SRS). The SRS is of size \(d\), used for circuits of up to \(\leq d\) gates. This universal SNARK has fully succinct verification and low prover run time \citep{PLONKcryptoeprint:2019/953}, compared to its predecessor Sonic \citep{SONIC10.1145/3319535.3339817}, which was the first universal and fully succinct SNARK. PLONK is based on constant round polynomial IOP and uses the polynomial commitment scheme based on \citet{Kate2010ConstantSizeCT}. It is presented as a non-interactive protocol, obtained through Fiat-Shamir transformation \citep{PLONKcryptoeprint:2019/953}. The trusted setup is universal and updatable: it can be used for the entire scheme and does not have to be produced for every problem (circuit), e.g., in Groth16. Also, the Kate commitment scheme can be replaced by any other polynomial commitment scheme, e.g., FRI (see chapter 5.2.3). 
Kate commitments use the elliptic curve generated points published in the public key after trusted setup, similar to Groth16 (see chapter 6.1 for detailed description), to commit to a polynomial of degree \(d\). The first \(d+1\) points are used to evaluate the polynomial at the respective coefficient. The underlying assumptions can be attributed to the \textit{Schwartz-Zippel lemma}.
\begin{align*}
    \text{\parbox{450pt}{\textbf{Schwartz-Zippel Lemma.} \textit{Let \(f(x)\) be a non-zero polynomial with degree \(d\) over \begin{math}\mathbb{F}^n\end{math}, then, for a randomly chosen \(r\), the probability of \(f(r) = 0\) is at most \(\frac{d}{n}\).}}}
\end{align*}
    
The \textit{Schwartz-Zippel lemma} proves that the polynomial evaluates to 0 at any point with high probability, if it evaluates to 0 at a given random \(r\). If two polynomials evaluate equally at \(r\), they are equal at every point with high probability. In a polynomial commitment scheme this suggestion is beneficial. The prover evaluates the polynomial at the random \(r\) chosen by the verifier, and sends it along with a proof. If the proof is valid, the verifier concludes that the result of the prover is also valid \citep{Kate2010ConstantSizeCT}.

The computation is first converted into an arithmetic circuit. Then, the arithmetic circuit is used to obtain a constraint system, similar to the R1CS from the previous chapter. Both have only one multiplication allowed per gate. However, PLONK only allows for one addition per gate, as long as it is not a constant. This constraint system also comes with copy constraints, which are used to transform the system into polynomials. The verification is executed using a polynomial commitment scheme (see Figure \ref{fig:plonk}).

\begin{figure}[hbt]
	\centering
		\includegraphics[width=0.8\textwidth]{Pictures/plonk_process.png}
	\caption{PLONK steps (simplified)}
	\label{fig:plonk}
\end{figure}

Similar to circuit-specific zk-SNARKs, e.g., Groth16, the computation has to be flattened for the protocol to process it. The problem is represented in an arithmetic circuit, consisting of gates that can be either representing an addition or a multiplication. Then, the arithmetic circuit is transformed into a constraint system representing the wires of the circuit. In analogy to circuit-specific zk-SNARKs, the constraint system is dependent on the number of gates (see chapter 6.1 for a simple example implementation). In PLONK, the constraint system is normalized into a specific form, which will be presented shortly, and is referred to as part of the public input in \citet{PLONKcryptoeprint:2019/953}. 
It is distinguished between constraints per gate and constraints across gates in the arithmetic circuit. The formalization system of gate constraints is the following:

\begin{align}
    Q_{L}a + Q_{R}b + Q_{O}c + Q_{M}ab + Q_C = 0
\end{align}

\(L, R, O\) represent the left, right and output gate wire. \(M, C\) stand for multiplication and constant. This standardized form allows representation of addition and multiplication. Setting \(Q_{M}ab = 0, \ Q_C=0, \ Q_{O}c = -1 \) and the rest to 1 will represent \(a + b - c = 0\). Setting \(Q_{L}a =1,\ Q_{L}b =1,\ Q_{O}c = -1\) and the rest to 0 will represent multiplication. Each gate is represented in the form presented in 5.1. Similar to the R1CS previously described, \(Q_{L}a, Q_{L}b, Q_{O}c, Q_{M}ab, Q_C\) can be expressed as vectors that hold the circuit structure. Again, in analogy with the R1CS, \(a, b, c\) can be expressed as vectors as well, which are the witness assignments. In correlation to chapter 5.2.1, these witness assignments might be private and only known to the prover.
In this constraint system, there are vectors \(Q_{L}, Q_{L}, Q_{O}, Q_{M}, Q_C, a, b, c\). Using the indices of these vectors as x, they can be transformed into polynomials of evaluation format. As an example, let us define one of the vectors in a circuit with 4 gates to illustrate the procedure. 

\begin{align}
    Q_L = (1,0,1,0) \text{ converts into the set of points } {(0,1), (1,0), (2,1), (3,0)}.
\end{align}

The set of points matches a degree 3 polynomial, which can be shown in a coordinate system (see Figure \ref{fig:examplepoly}). Through Lagrange Interpolation, the concrete polynomial in coefficient form can be calculated. 
\begin{figure}[hbt]
	\centering
	\includegraphics[width=0.6\textwidth]{Pictures/example_polynomial.png}
	\caption{Example polynomial \(Q_{L}(x)\) evaluated at the points in 5.2. Lagrange Interpolation yields coefficient form \(Q_{L}(x) = - 0.6667x^3 + 3x^2 - 3.3333x + 1\)}
	\label{fig:examplepoly}
\end{figure}

This procedure is applied to all Q-vectors and the vectors a, b, c to transform them from constants into polynomials. The corresponding function obtained is
\begin{align}
    f(x) = Q_{L}(x)a(x) + Q_{R}(x)b(x) + Q_{O}(x)c(x) + Q_{M}(x)a(x)b(x) + Q_{C}(x) = 0\\
    f(x) = Z(x)H(x)
\end{align}

This will allow to compress as much information as possible into a single source, i.e., a polynomial. 

Besides the constraints as per gate, there are also constraints that hold across gates, e.g., the output of a gate can be equal to the input on another gate. It is necessary to represent them too, in order to translate the whole problem into the scheme. These copy constraints can lie within one vector, e.g., if \(b0 = b2\), or between multiple vectors, e.g., \(c0 = a3 = b1\). In the first case, the indices of the vectors are exchanged to create a permutation function \(\sigma(i)\). In the second case, the vectors are combined into one long vector, before the permutation function is obtained. Once all copy constraints are generated as lists of permuted indices, they are transformed into polynomials of the permuted gate indices. The resulting polynomials are
\begin{align}
    \sigma_{a}(x), \sigma_{b}(x), \sigma_{c}(x)
\end{align}

A proof is generated by using these permutations and computing the values of all the gates to create the polynomials of \(a(x), b(x), c(x)\). Now, an accumulator \(p(x)\) will be designed in order to represent all coordinates in the set of points between the vectors \(a, b, c\). It proves the copy constraints as follows.
\begin{align}
    p(x+1) = p(x) * (v_1 + X(x) + v_2 * Y(X))
\end{align}
\(v_1, v_2\) are random values and \(X, Y\) are polynomials per vector representing the x and y coordinates. For every copy constraint, e.g., \(a1=a3\) or \(b4=c1\), there will be a \(X(x)\) representing the x coordinates \(a,b ,c\) (note, these are the indices introduced (5.2). \(X'(x)\) will be a polynomial representing the indices flips in the copy constraint. Every vector will be represented in \(p_{a}(n), p_{b}(n), p_{c}(n)\) and \(p'_{a}(n), p'_{b}(n), p'_{c}(n)\). Instead of checking each copy constraint within the same vector individually, the following multiplication check is performed to also check copy constraints across vectors at the same time:
\begin{align}
    p_{a}(n) * p_{b}(n) * p_{c}(n) = p'_{a}(n) * p'_{b}(n) * p'_{c}(n)
\end{align}

In practice, the permutation accumulators are not expressed in dependency of vector size \(n\), but through high order roots of unity, whereby the field elements satisfy \(\omega^n = 1\). Also, all values are expressed as elements within a finite field of prime order \(p\), similar to the circuit specific zk-SNARKs. The X coordinates are not expressed as indices dependent of vector size \(n\), but with \(omega\) and some random element \(g\) in the field.

\subsubsection{Summary}
After transforming all the constraints intro sets of polynomials, the following checks need to be performed to verify the proofs \citep{PLONKcryptoeprint:2019/953, buterinplonk}. Through the polynomial commitment scheme based on \citep{Kate2010ConstantSizeCT}, these checks can be verified. Elliptic curve points are generated at random, similar to circuit based zk-SNARKS, and used to evaluate the polynomials. Elliptic curve pairings allow to check whether the equations hold true without any of the generated points, or polynomials to be revealed. 
The equation in (5.3) is the main equation of the circuit that needs to be checked. Then, in total, there are six permutation accumulator functions for the witness assignment vectors and their copy constraints. The equation check in (5.4) is resulting into six rounds:
\begin{align}
\end{align}
\begin{align*}
    P_{a}(\omega x) - P_{a}(x)(v_1 + x + v_2a(x)) = Z(x)H_{1}(x) \\
    P_{a'}(\omega x) - P_{a'}(x)(v_1 + \sigma_{a}(x) + v_2a(x)) = Z(x)H_{2}(x) \\
    P_{b}(\omega x) - P_{b}(x)(v_1 + gx + v_2b(x)) = Z(x)H_{3}(x) \\
    P_{b'}(\omega x) - P_{b'}(x)(v_1 + \sigma_{b}(x) + v_2b(x)) = Z(x)H_{4}(x) \\
    P_{c}(\omega x) - P_{c}(x)(v_1 + g^{2}x + v_2c(x)) = Z(x)H_{5}(x) \\
    P_{c'}(\omega x) - P_{c'}(x)(v_1 + \sigma_{c}(x) + v_2c(x)) = Z(x)H_{6}(x)
\end{align*}

There are constraints for the accumulator to be checked which result from (5.7):
\begin{align}
    P_{a}(1) = P_{b}(1) = P_{c}(1) = P_{a'}(1) = P_{b'}(1) = P_{c'}(1) = 1 \\
    P_{a}(\omega^n)P_{b}(\omega^n)P_{c}(\omega^n) = P_{a'}(\omega^n)P_{b'}(\omega^n)P_{c'}(\omega^n)
\end{align}

The only program specific polynomials that need to be computed upfront by prover and verifier are the \(Q\)-polynomials from the circuit and the \(\sigma\)-permutation polynomials. The verifier algorithm only stores commitments to these polynomials. The user inputs are the witness assignments \(a(x), b(x), c(x)\), the accumulators \(P\) from above and the different \(H\) for every round. Although the verification is efficient, the proof size is still an area of improvement. For the implementation of plonk in circom and snarkjs, see chapter 6.2.2.

\subsection{FRI-STARK}
\begin{comment}
with FRI-based polynomial commitment
-does not use R1CS
-fri starks
-Salleras et al 2021
\end{comment}

\subsection{Bulletproofs}
\begin{comment}
with discrete-log based polynomial commitment
- uses R1CS too
- they are non-interactive,
- they are zk arguments of knowledge
- Godden et al bulletproofs
- Chung et. al bulletproof+
- Deng at al history of bulletproofs
-Salleras et al 2021
main protocols: Lipmaa, Boudot, then Groth and Coutueau, then Hybrid from Kim Lee 2019
Deng et al 2022: history of range proofs, cuproof as example
-Kim Lee 2019: overview on range proof protocols
\end{comment}

\section{Application Domains and Use Cases}
Zero-knowledge proofs find broader implementation due to increasing interest in decentralized applications in recent years. However, despite successful research efforts in distributed ledger technology, there is only limited adoption of blockchain-based solutions outside the financial sector. One of the main challenges, pointed out by \citet{SedlmeirTransparencyChallenge}, is the need to make sensitive data visible. The high transparency blockchain offers collides with the need to preserve privacy and allow for restricted visibility. \citet{Godden} describe it as increasing consciousness to preserve data confidentiality and ownership, which leads to the development of privacy enhancing technologies. The literature review (SLR approach is described in chapter 4) shows that general purpose zero-knowledge proofs find implementation as essential elements in domains with enhanced privacy preserving efforts. ZKP implementations can be categorized into the following application domains: \textit{Identity Management}, \textit{Data Sharing and Traceability}, and, \textit{System Scaling and Protocol Enhancements}. Use cases from these application domains are clustered according to the problem domain they are attempting to solve (Table \ref{tab:domains}): (1) \textit{Electronic Voting and Government}, (2) \textit{Electronic Auctions}, (3) \textit{Data Queries and Traceability}, (4) \textit{Electronic Healthcare}, (5) \textit{Cloud Security}, and (6) \textit{Scaling and Performance}.

\setlength{\tabcolsep}{2ex}
\renewcommand{\arraystretch}{1.5}%
\begin{table}[ht]
	\centering
	    \caption{Selective Literature Overview - Problem Domains}
		\begin{tabular}{| m{0.02\linewidth} | m{0.3\linewidth} | m{0.4\linewidth} |}
		\hline
		\textbf{} & \textbf{Problem Domain} & \textbf{Literature} \\ \hline
            1 & Electronic Voting and \newline Government & \citet{Bansod, Guo, Querejeta} \\  \hline
            2 & Electronic Auctions & \citet{LiXue, WangZhaoMu} \\ \hline 
            3 & Data Queries and \newline Traceability & \citet{Godden, XueWang}  \\  \hline
            4 & Electronic Healthcare & \citet{LuongPark, ZHENG, WangEtAl, Huangetal} \\  \hline 
            5 & Cloud Security & \citet{LiuWangPengXing, Major, Munivel, Kanagamani} \\  \hline 
            6 & Scaling and Performance &  \\  \hline 
	\end{tabular}
\label{tab:domains}
\end{table}

\begin{comment}
sedlmeier et al 2022:
- despite successfull research in blockchain based applications, little adoption in businesses beyond financial sector
- one challenge for adoption: visbility of sensitive data
- excessive transparency in blockchain applications and need for privacy preservation of sensitive data, restricted visibility
- ZKPs as one possible solution
- Godden et al describe it as increasing conciousness about data ownership and privacy led to research in privacy enhancing technologies (PETs)
- sources from Zhang et al 2021 PipeZK overview of the different Anwendungsbereiche of ZKPs-->what is verifiable outsourcing as application example?
-SLR showed these main application domains for ZKP: Identity Management, Data Sharing/Traceability, Blockchain Scaling
- with the following use cases:
-e-government:e-voting/digital election
-healthcare: purchasing medical insurance contracts, patient data exchange/access
-e-auction/bidding systems
-cloud computing: zk port knocking, data deduplication, authentication schemes for cloud servers
-blockchain scaling: decrease comp cost for tx verification, zkRollups, zkSync
-querying: link traversal, zk-SQL SPARQL queries, zk based traceability systems


- ZKPs belong to verifiable computation, succinct blockchain Simunic et al 2022

- Maller et al 2019 Sonic: verifiable outsourcing computation example paper
\end{comment}

\subsection{Electronic Voting and Government}
Motivated by recent developments in personal data protection laws, \citet{Bansod} present a governmental architecture based on self sovereign identity to cope with the increasing demand to protect online transactions of personal information. In this generalized scenario, users request a decentralized identifier from their digital, governmental issuer, which is going to ask for a piece of personally identifiable information that is needed for administrative services, e.g., birth date, address, tax identifier etc. The user only provides a zero-knowledge proof enabled identity, e.g., the proof of being the person the user claims to be, and receives identity certificates from the e-government issuer. These certificates are encrypted and stored in a database and on-chain. The user, when in need of a service, e.g., renew a driver's license, can request from service providers and provide the certificates that are required. The service provider can verify the hash values and digital signatures on the certifications and provide the service once it holds true.

Current privacy preserving efforts in governmental biometric identification did not leave traditional cryptography yet. \citet{Guo} propose a novel approach to decrease governmental expenses and enable scaling of the systems. A zk-SNARKs based approach is presented which reduces efficiency and eliminates fingerprint template disclosure. First, the fingerprint features are extracted by calculating the Euclidean distance of points collected. If the sizes of the distances match a certain threshold, the authentication is passed. This computation is being converted into a zk-SNARK friendly, polynomial computation with specific constraints that resulted from the first step. Then, this computation is transferred into a circuit, then to a R1CS, and later into a QAP with three polynomial matrices. The trusted setup generates proving and verifying key via CRS. The prover algorithm creates the prove with proving key and private witness and thee verifier algorithm can perform the verification through elliptic curve pairing. The Groth16 algorithm was used as basis to apply it to the biometric authentication example, which leaves improvement for disposing the trusted setup requirement and QAP computational time.

There are various implementation of electronic voting schemes using zero-knowledge proofs. \citet{Querejeta} introduce a first verifiable re-voting scheme with linear complexity. Voters are authenticated and can vote multiple times from any device in zero-knowledge. Only the latest vote is counted. Adversaries cannot obtain any further information about the vote than the final result used for the election, even if they possess unbounded computational power. This can be achieved by introducing a honest certification authority. In the pre-election phase, the voting server generates voter identities at random, which are being encrypted using the public key of the voting server and published to the bulletin board. The voters can receive such an anonymous identity and prove they are legitimate. Every time a vote needs to be made, the voting server has to create the voting identities, and the voter signs with their key, so that multiple votes can be allocated to a single voter. The voting server verifies the vote and publishes it as ballot to the bulletin board. The voter can verify that the vote was put forward. \citet{Querejeta} included dummy votes into the scheme to prevent adversary attacks. The voting server casts dummy votes to hide the real number of votes in the election. Finally, in the tallying phase, the tellers proceed and decrypt the votes in a verifiable and private manner. Interactive zero-knowledge proofs under the random oracle assumption are used to simulate proofs, and turned into non-interactive proofs via Fiat-Shamir transformation.

\subsection{Electronic Auctions}
Zero-knowledge proof implementation efforts for bid e-auction schemes aim at preventing the exposure of bid information and bid price, as well as protecting the bidder's identity. \citet{LiXue} show that the goal can be met, while in addition, the use of an auctioneer no longer becomes necessary. The latest bid price is hidden through Pedersen's commitment scheme, and the ZKP algorithm makes use of Bulletproofs so the bidder can prove that the new bidding price is higher than the previous price. Every participating bidder can verify it. The sealed-bid auction smart contract interacts with the blockchain by publishing the commitments and initializing the bidding process for the owner. The information about goods is also encrypted in the process, as well as the price and winning bidder. All participants verified the price and the winner without obtaining knowledge about them. This decentralized e-auction scheme ensures sealing and fairness, while protecting the privacy of the participants. Although there is no cost involved to engage a third party auctioneer, the productive cost of running such a platform remain to be explored.

Similar implementations without the use of public blockchain structures are found in \citet{WangZhaoMu}, who make use of Hyperledger. Members of the consortium are able to make use of private channels for secure communication, while effectively realizing the overall use of smart contracts and transaction privacy. In the endorsement process, the bidder initiates the transaction and the client submits it to the endorsing peer. The endorsing peer simulates the transaction proposal and verifies it. The transaction is initiated in the ledger and returned to the platform. In the ordering process, the platform performs a combination of transaction and endorsement to be sent to the ordering peer. This peer puts the transactions into transaction blocks for every channel and delivers them to the committing peer. In the verification process, the committing peer verifies the transactions in the blocks, submits them to the ledger and send notification to the platform. Zero-knowledge proofs are used to manage the participants' identity and enable only authorized users of sending requests to the client. Despite the promising architecture of Hyperledger Fabric, the practical adoptions can get very complex, depending on the use cases. This implementation requires more computational resources, in combination with the use of zero-knowledge proofs in particular.

\subsection{Data Queries and Traceability}
Protected data sharing, securing data queries and corresponding outputs have become increasingly important during the COVID-19 pandemic. \citet{Godden} propose Circuitree, a ZKP-based tool that can be used to exchange verified pieces of information in zero-knowledge, and implement the example of digital COVID certificates. The underlying query language is Datalog. A Datalog verifier is presented, which provides verification about each logical querying step in the reasoning process. This zero-knowledge Datalog engine is fed with domain-specific sets of rules, e.g., vaccination, test and recovery data. The verifier, e.g., the restaurant with specific entry rules during the COVID-19 pandemic, creates this rule set in Circuitree. Thee prover, e.g., a vaccinated person, declares their facts to Circuitree and signs them. This builds the tree-like R1CS structure, which is converted into a bulletproof based system. The prover can create the proof by querying the Datalog engine and, e.g., provide it to the restaurant for admission.

The problem of missing privacy preserving traceability for product development and supply chain data is being addressed in \citet{XueWang}. Parties in the industrial product development sphere are enabled to track the development history of products without trusting each other. There are three main layers to the newly-proposed process architecture. The traceability application layer authenticates data owners and interacts with a third party, the traceability agent. The data privacy layer, which contains the zero-knowledge proof implementation, is responsible for generating traceability features in zero-knowledge and interacting with the traceability data providers, e.g., another partner in the production process inquiring to trace some data on a specific product. The data traceability request is being processed to create a proof via a smart contract. The data owner acts as verifier of this proof. The traceability features and use of the smart contract happens in the physical data layer and is mainly performed by a third party. The traceability process can be entirely monitored publicly, which solves the transparency problem in the industry, while preserving the traceability inquiries and adapting to the low-trust environment of the participants.

\begin{comment}
Fang, Near, Darias, Zhang 2021
more papers on machine learning code verification ZKPs-->look in ACM pdf
- Xue and Wang: traceability use case applicable to pred. maintenance problem
-link traversal, SQL
-zero knowledge static program analysis: proof that a secret code is correct (Fang, Near, Darias, Zhang 2021)
\end{comment}

\subsection{Electronic Healthcare}
Patient monitoring nowadays appears to be increasingly remote, i.e., health data collection happens via medical devices. \citet{LuongPark} make use of zk-SNARKs to enable patient medical data sharing between medical devices and health service providers. The main interactions happen between the patient, medical device and health service provider. The health service provider is responsible to collect patient data from the medical device, analyze it and respond with adapted features and enhanced functions in the medical device provided. The patients initialize the process by using their public address in the blockchain system and create signed hashes. The provider uses these hashes to create an arithmetic circuit and initiate the zk-SNARKs protocol. Parameters and the zk-proof are created via a smart contract. Patients can use the smart contract to authenticate themselves to add additional information to it, e.g., the device. Through a secure key exchange algorithm, the health service provider can share encrypted patient data with medical devices via a secure, zero-knowledge communication channel. The underlying zk-SNARKs used is Pinocchio \citep{Pinocchio}, implemented via Zokrates \citep{ZoKrates}. Even though the proof generation takes long and the system is not suitable for mobile devices due to the nature of the tools used, it solves current problems due to lack of anonymous and secure patient data sharing between medical devices and health service providers, e.g., adversary attacks and unauthorized disclosure of health data.

Patient health data privacy protection efforts also reach the medical insurance purchase and claim process, as underlined by \citet{ZHENG} and \citet{WangEtAl}. Health data is shared in a secure channel and the patient identity is disclosed to a minimum using non-interactive zero-knowledge proofs. Patients provide their health data in a regulated and authenticated manner via smart contracts. Hereby, the hospital acts as fully trusted data generator, interacting with the smart contract. Patients can obtain their medical data from the hospital, which will provide a unique identity for them. The insurance companies publish their restrictions and requirements, e.g., for purchasing a medical insurance, via smart contracts on-chain. These requirements are used by the hospital to build the constraint system and circuit, and finally to produce a proof, alongside with the encrypted medical data and identity of the patient. The medical insurance company can verify this proof and the smart contract can initiate the process of purchasing or claiming between patient and insurance company, without compromising patient identity and sensitive health data insights. The validity of the payment transaction is verified by other peers in the blockchain.

In contrast to the increasing consciousness about data protection and excessive amounts of patient data available, there is the need to process medical data effectively to enable research and health care. Implementations of \citet{Huangetal} try to leverage these opposites by implementing secure medical data sharing between patients, health care providers, research institutions and semi-trusted servers on cloud. It is achieved through making use of a private blockchain in Hyperledger Fabric, in combination with zk-SNARKS. Research institutions put their requirements for medical data, e.g., study inclusion criteria, into an arithmetic circuit and publish a proof. The patient encrypts their medical data on their own or can authorize their health care provider, e.g., hospitals. The encrypted and signed medical data is sent to the semi-trusted cloud server and broadcasts the encrypted data on-chain. Whenever patients decide to share their medical data with research institutions, another proof has to be created to show that the medical data matches to the research criteria of the research institution. It is achieves via smart contracts and the proof algorithm specified there. It also verifies this proof, and upon success, the patient is able to generate the re-encryption key which is used in conversion to the public key of the specific research institution. The cloud server receives this re-encryption key and also signs it on top with its public key. This enables the research institution the decrypt the medical data, while the semi-trusted cloud server cannot obtain any further knowledge. It is captured in a transaction, that can be verified by participating nodes in the network via consensus.

\subsection{Cloud Security}
Robust authentication schemes for client and user authentication on cloud servers experience enhancement due to the increasing practical applicability of zero-knowledge proofs. Implementation efforts of \citet{LiuWangPengXing} show how a center-less and biometric-based single sign on across cloud services can work. The user gets registered in the registration center, which does not participate another time in the authentication procedure, which removes centralization vulnerabilities. There is a token service provider, which generates a zero-knowledge token for the user which is used across multiple cloud services. The underlying technique is based on circuit-specific zkSNARKS, e.g., Groth16. An elliptic curve over a finite field with generator points are used and a common reference string is issued, whereby the token service provider performs the setup phase. The user adds secret values, i.e., user identity, password and biometric information as secret values. The token service provider registers the user and delivers a token, without learning the decrypted user's secret values. The cloud service provider also registers via the token service provider. Each registration ends with a zero-knowledge token provision. Both, user and cloud service provider, verify eachother, whereby a specific session key is generated. This session key is used to authenticate the user on other cloud service providers.  

In \citet{Major}, prototype I applies zero-knowledge proofs for lightweight and private client server authentication, based on port knocking. Port knocking is widely used as authentication mechanism between clients and firewalls, allowing for a channel between them within an untrusted network, e.g., the internet. The client is authenticated by the host without open ports and attacks are difficult because the machine's function as a server is hidden. Non-interactive zero-knowledge proofs are used in the first prototype to work towards the goal of hide any further knowledge from sniffing traffic or eavesdropping. In the setup phase, profile files for client and server are created, whereby only the client has a secret private key in the file, which is randomly selected. Furthermore, the files contain the parameters for the ZKP, a private hash key, the server port number and the command that is to be run upon successful authentication, e.g., to mount a app service. The client creates the proof, which is treated like the knock, and transmit it to the port knocking server. The server parses the traffic, inspects it and checks whether ZKP criteria are met, e.g., that the server port specified by the client matches. If the checks are met, then the protocol is executes further to perform the computational verification through bilinear pairings. If the verification passes successfully, the client-specific command can be executed.

Password attacks have increased, especially observing the rising usage of cloud storage service for mobile devices. In \citet{Munivel}, this threat is analyzed and a new authentication scheme is proposed, using zero-knowledge proofs for mobile cloud storage authentication. The client server provides a unique in-browser mask for entering user identifier and password. In the background, these values are hashed and do not leave the browser as entered. With the public key of the user and a random value, which is element of a cyclic group of prime order, the user's algorithm calculates a proof consisting of a random token, the password hash hidden by some generator from the cyclic group and the random value, and the public key of the user. The server calculates an own verifying key, and, i.e., similarly to the circuit-specific zk-SNARKS verify algorithm, can check whether the proving information sent by the user matches. The server can do this, because the server has access to the random token, user public key and the group element generator, which are public elements. Neither the server, nor an adversary, can obtain the user password and receive access to the cloud storage.

Apart from research work focusing on cloud authentication, there are implementation efforts to make use of zero-knowledge proofs for storing only one single copy of the same data on cloud servers, i.e., data deduplication efforts. Motivated by mass data storage outsourcing to third party cloud computing providers, data deduplication and dynamic ownership are promising areas of development. On the assumption that the cloud server is honest, but curious, \citet{Kanagamani} propose a data deduplication scheme using in-line block matching and interactive zero-knowledge proofs. The cloud server performs the deduplication check by verifying the proof of ownership of the file and checking whether a copy of the file is already stored, while learning no further information about the ownership and the file. Before the initial upload, the file should been hashed already. The server registers users and provides them with a public key and secret key. By applying another hashing algorithm on the file hash, the encryption key is obtained. Further, this key is hashed once more to obtain the tag. The tag is referred to by the cloud server to check whether a subsequently uploaded file already exists. The user chooses a random encryption key, which is different and encrypts the encryption key to obtain a ciphertext. The tag, ciphertext, user identifier and the proof is stored in the cloud server. The proof is obtained by combining in-line block matching with zero knowledge proof generation. The file is divided into blocks and each block is used to perform a exponential equation with some primes \(a\) and calculated in modular arithmetic with another prime \(mod \ b\). All computations are performed using a multiplicative cyclic group \begin{math} \mathbb{G} \end{math} with prime order \(t\). The result of each calculation forms a sequence of individual block proofs, which are the file proof. The server receives the proof, together with the data described above, and checks if the tag matches any other tag of a file stored previously. For verification, group keys are generated, which are used to perform bilinear pairing checks, i.e., similar to zk-SNARKS, e.g., Groth16. The public key of the file is taken, alongside random prime numbers to obtain the group key.The ciphertext is encrypted with the group key by the server and stored together with the ownership information. Each time the ownership changes, the server uses the existing proofs to send challenges to the allegedly new owner. The new owner responds by creating secret values from the challenge received. Once the server can verify the proof, secret and response, another group key is generated and used to encrypt the ciphertext of the file. The ownership information can be changed. The server did not learn anything more than there has been a change in ownership and that the file only exists once. 

\subsection{Scaling and Performance}
\begin{comment}
- yang yang et al
- salleras et al
- deng et al
-Soonhyeong 2021: better verification with EVM to verify non-maliciousness of blocks (zKSNARKS used)
- zk Roll ups, more sources needed
- Xu Chen(2021): new algorithms for zero knowledge set membership for ZK-scaling, based and compared to zkSync
Yang, Weng, Sarkar, Wang: memory effieciency enahncements
- Shi et al (2022): Schnorr ZKP used to proof user identity-->a bit old fashioned ZKP used maybe??
\end{comment}

\section{Challenges}
- challenges can go into evaluation/comparisons of the 4 zkps
\subsection{Cost}
-Zhang et al 2021 PipeZK time and cost challenges of zKSNARK
-effieciency of ZKP: computation depends on field size-->there is a need for memory efficient ZKPs: wolverine and mac'n'cheese can do it better, but QuickSilver better protocol for large circuits (Yang, Weng, Sarkar, Wang 2021), Dittmer et al (2022) outperform Quicksilver then (most current best performing memory based protocol)
-elements from Sedlmeier Völter Strüker (2021): take the cost of Groth16

\subsection{Trust Assumption}
-Huang et al 2020 semi trusted proxy server, their assumptions
-trusted set up for zk Snarks

\subsection{Quantum Computing Threats}
1. Why are quantum verifiers a threat? (maybe Katz et al 2018)
2. different aspects of solution examples
-Deng et al 
- Xie Yang 2019: quantum secure CRS, good arguments of other shortcomings of ZKP systems, interactive and non interactive quantum zero knowledge proof systems
- Vidick Zhang 2020: describe the three problems that there are protocols developed for (big umbrella problem: quantum verification problem), good definitions from the quantum world
- also include Watrous(2009)-->coming from Vidick Zhang 2020
-Lyubashevsky et al (2020): new lattice based ZKP algorithm which is the fastest and smallest proof size for small int addition and multiplication 

\section{Evaluation Methods}
-Huang et al 2020: 1)privacy preserving and security: confidentiality, availability, integrity, privacy-preserving, traceability, single point of failure 2) performance evaluation: computing cost, number of startup nodes, privacy protection, time to generate NIZK keypair, NIZK proof, verify NIZK proof
-Zhang et al 2021 PipeZK proof generation enhancement
- Maller et al 2019 Sonic: better structured SRS in linear size to speed up proof verification

\subsection{Performance Analysis}
-Liu et al, Zheng at al: semihonest model evaluation topics-> computational complexity, communication complexity, experimental evaluation
- Liu Wang Peng Xing 2019: remote authentication for mobile cloud computing, Real-Or-Random model and BAN logic for security evaluation REGARDING different attacks
- Zhang et al 2021 PikeZK: new hardware accelerator to boost comp time for proof generation

\subsection{Sustainability}
-maybe too little literature for an own sub chapter
-Simunic et al mention need for more sustainable solutions in blockchain privacy preserving through ZKPs
-elements from Sedlmeier Völter Strüker (2021): ZKPs are more sustainable 
    \clearpage{\pagestyle{empty}\cleardoublepage}
    \chapter{Implementation and Results}
The following sub chapters present the results of this master's thesis. First, requirements are summarized. Subsequently, the example computation of a Groth16 proof and verification is described in detail. The first artifact, the zero-knowledge decentralized application (zk-DApp) is demonstrated. The second artifact, an architecture proposal for zero-knowledge data structure of spare part certification and meta data information is introduced. Ultimately, artifacts are evaluated and a future outlook is provided.

\section{Summary of Requirements}
The requirements positioned in previous chapters (4 and 5) can be summarized as follows: 

\begin{enumerate}
    \item Zero-knowledge proof systems are identified to be a disruptive technology in the domain of blockchain-based research and development. One project requirement is to \textbf{accumulate knowledge and current research outcomes} in this field to extend expertise within RAPADO \citep{ZedelJ}. As an outcome to the extensive literature review, the Groth16 algorithm is introduced in detail by using an example calculation problem (6.2). In the step-wise computation, the problem is split and transformed to an arithmetic circuit, arriving at a R1CS. From this, the QAP is calculated. Additionally, tools, e.g, homomorphic hiding and elliptic curve pairing are introduced. Ultimately, following the Groth16 protocol, the key and proof generation, as well as verification mechanisms are illustrated.
    \item The requirement of \textbf{constructing a trade-off between privacy, confidentiality and transparency} arises from preliminary work in the project and at the department of information systems \citep{FornaconFrank, ZedelJ, semesterproject}. The use of blockchain-based solutions enables great transparency at the cost of confidentiality. With the constantly increasing awareness for data privacy, combined with trust issues and regulatory data confidentiality requirements in the industry, zero-knowledge proof systems have to be explored and utilized to secure adoption of project results and products in the future. The first minimal viable product (MVP) of the zk-DApp for MRO data attestations contributes to this goal and satisfies this requirement. The zk-DApp is organized as follows: the backend directory stores the json input schema, the solidity smart contracts, the circom circuit and corresponding generated files (6.2). The lib directory stores shared files between backend and frontend, as well as hashing functions used, e.g, in the circuit code. The UI directory finds the correct schema to be used and also stores a copy of the final protocol transcript key and the witness generating file. It also stores the frontend code for the landing page and all corresponding pages to submit, attest to, and verify MRO data.
    \item Current paper-based MRO documentation of aviation spare parts largely sets the requirement for \textbf{fraud-preventive verification mechanisms, enabled though effective data formats and digitization of spare part documents}. Fraud-preventive verification is covered via the zk-DApp. However, current data digitization efforts for spare part documentation need to be further addressed: the zero-knowledge data structure architecture enables consistent and temper-proof data storage via merkle trees, and proofs specific memberships, e.g., mechanics who worked at a specific part at a given time, in zero-knowledge.
\end{enumerate}

\begin{comment}
-research disruptive technologies and creating knowledge in the project: calculation example Groth16
-privacy/confidentiality vs transparency : zk-DApp
-need for fraud-preventive verification mechanism enabled through effective data formats digitization of aviation parts and document data: architecture



Notizen:

In den Ausblick der zk-dapp
-für einen part die schwellenwerte finden
-time since last service
-validation network as attester—>Übergang zum 2.artefakt 

-schwellenwerte ausgelesen werden
-aber auch die attester müssen ausgelesen werden

—>release certificate: Behörde
—> shop report: MRO Betrieb

-MRO KPIs: für den Handel, aus den MRO Daten ergeben, aus Privatsphäre/Transparenz requirement, ohne dass man Einsicht ins gesamte shop report geben will
\end{comment}

\section{Groth16 proof and verification}
Let us use an example to illustrate the underlying mathematical methods that are applied in zk-SNARK. The example calculation will use the knowledge of the coefficient assumption for simplification. In practice, the FFT is applied. First, the arithmetic circuit is transformed into a R1CS. The R1CS is used to obtain the QAP. Homomorphic hiding, elliptic curves and pairing-based cryptography are introduced in more detail and put in context for the next steps of the calculation. Finally, the Groth16 protocol is introduced: First, the key generation steps are explained. Second, the proof is generated. Lastly, the verification steps are illustrated. Formal definitions are found in 5.2.

Say we want to prove we know a secret x so that

\[x^3 + x + 5 = 35\]

In this case, our secret is x = 3.
In practice, we would use hiding and modular arithmetic instead of real numbers and calculations since these are easy to forge and find solutions to, which will make the proof useless. For R1CS and QAP we will proceed with real numbers to show the underlying mechanisms. The following will demonstrate how any computation that needs to be proven can be converted into polynomial format.

\subsubsection{Arriving at a R1CS}

A rank-1 constraint system is a mathematical format to help us reduce our problem into a less complex computational problem. First, we flatten the equation by writing a short program that would break down the different steps to solve the equation.

\begin{enumerate}
    \item \(sum1 = x * x\)
    \item \(y = sum1 * x\)
    \item \(sum2 = y + x\)
    \item \(out = sum2 + 5\)
\end{enumerate}

As shown above, we arrive at an arithmetic circuit with 4 gates and the solution variables
\[x = 3, y = 27, sum1 = 9, sum2 = 30, out = 35.\]

From this, we can construct the solution vector \(s\) , which has to start with a dummy variable of value 1, which we call \textit{one}.
Now, the solution vector \(s\) is
\begin{align}
    \Vec{s} &= \begin{pmatrix}
     one \\ x \\ out \\ sum1 \\ y \\ sum2
\end{pmatrix}
\end{align}
Each gate will be represented so that
\begin{align}
     \Vec{s}\cdot\Vec{a_i} * \Vec{s}\cdot\Vec{b_i} - \Vec{s}\cdot\Vec{c_i} = 0
\end{align}

Let us go through every gate and assign the values for a, b and c.
For the first gate \(sum1 = x*x\), the values of a, b and c are assigned as follows:
\begin{align*}
    a_1 &=\begin{bmatrix}
        0 & 1 & 0 & 0 & 0 & 0
    \end{bmatrix}
\end{align*}
\begin{align*}
    b_1&=\begin{bmatrix}
        0 & 1 & 0 & 0 & 0 & 0 
    \end{bmatrix}
\end{align*}
\begin{align*}
    c_1&=\begin{bmatrix}
        0 & 0 & 0 & 1 & 0 & 0
    \end{bmatrix}
\end{align*}

This is correct, because the dot product of s and a, multiplied by the dot product of a and b, subtracted by the dot product of s and c is 0 (6.2).

This procedure is applied to every gate. Let us show more complex gates to underline the calculation. For example, the third gate and the fourth gate. The third gate \(sum2=y+x\) could be approached as the first gate, by setting the variables in the equation to 1. However, this would not fulfill the equation shown in (6.2). Therefore, the correct values for \(a_3, b_3 \text{ and }c_3\) are
\begin{align*}
    a_3 &=\begin{bmatrix}
        0 & 1 & 0 & 0 & 0 & 0
    \end{bmatrix}
\end{align*}
\begin{align*}
    b_3&=\begin{bmatrix}
        1 & 0 & 0 & 0 & 0 & 0 
    \end{bmatrix}
\end{align*}
\begin{align*}
    c_3&=\begin{bmatrix}
        0 & 0 & 0 & 0 & 0 & 1
    \end{bmatrix}
\end{align*}

Here, we make use of the dummy vector \textit{one}, so that we can arrive at 
\[30 * 1 - 30 = 0.\]

The fourth gate \(out=sum2+5\) also has to be approached by holding true to the dot product equation in (6.2) as well. We have to make use of the dummy vector once again. Setting the values of \(one, sum2 \text{ and } out\)  to 1 will give us the following incorrect solution:
\begin{align*}
     \Vec{s}\cdot\Vec{a_4} * \Vec{s}\cdot\Vec{b_4} - \Vec{s}\cdot\Vec{c_4} \neq 0.
\end{align*}
The calculation shows \(30 * 1 - 35 \neq 0\), which means we need to add 5, so that the dot product of vector \(s \text{ and } a\) adds up to 35. Therefore, the values of \(a_4, b_4 \text{ and }c_4\) are as follows:
\begin{align*}
    a &=\begin{bmatrix}
        5 & 0 & 0 & 0 & 0 & 1
    \end{bmatrix}
\end{align*}
\begin{align*}
    b&=\begin{bmatrix}
        1 & 0 & 0 & 0 & 0 & 0 
    \end{bmatrix}
\end{align*}
\begin{align*}
    c&=\begin{bmatrix}
        0 & 0 & 1 & 0 & 0 & 0
    \end{bmatrix}
\end{align*}

By combining our results into matrices, we can set up the corresponding R1CS:

\begin{align}
A&=\begin{pmatrix}
    0 & 1 & 0 & 0 & 0 & 0 \\
    0 & 0 & 0 & 1 & 0 & 0 \\
    0 & 1 & 0 & 0 & 1 & 0 \\
    5 & 0 & 0 & 0 & 0 & 1
\end{pmatrix}
\end{align}
\begin{align*}
B&=\begin{pmatrix}
    0 & 1 & 0 & 0 & 0 & 0 \\
    0 & 1 & 0 & 0 & 0 & 0 \\
    1 & 1 & 0 & 0 & 0 & 0 \\
    1 & 0 & 0 & 0 & 0 & 0
\end{pmatrix}
\end{align*}
\begin{align*}
C&=\begin{pmatrix}
    0 & 0 & 0 & 1 & 0 & 0 \\
    0 & 0 & 0 & 0 & 1 & 0 \\
    0 & 0 & 0 & 0 & 0 & 0 \\
    0 & 0 & 1 & 0 & 0 & 0
\end{pmatrix}
\end{align*}

\subsubsection{From R1CS to QAP}

The R1CS shows three matrices \(A, B \text{ and }C\) representing the four gates each of length six. It is transformed into a QAP by expressing polynomials as sums of Lagrange Interpolation (chapter 5.2.2). This results in three sets of polynomials \(A_i(X), B_i(X) \text{ and }C_i(X)\) each consisting of six polynomials of degree three. In the following, we will summarize the reason of setting up a QAP instead of continuing with the R1CS. Lagrange Interpolation allows us to come up with polynomial coefficients, which represent each gate, when evaluated at an X in the range of number of constraints (gates). With X = 1, the Lagrange Interpolation can be explained quite well, because it means that we can just add up the coefficients of the polynomials in \(A_i(X), B_i(X) \text{ and }C_i(X)\). Each set of polynomial is built so that evaluated at a gate X, whereby X has to be in the range of the number of gates (constraints), will deliver the specific value of X and 0 for the other values in that specific range.\newline
The QAP for our example:
\begin{align*}
    A_i(X) \\
    \begin{bmatrix}
        -5.0 & 9.166 & -5.0 & 0.833 \\
        8.0 & -11.33 & 5.0 & -0.666 \\
        0.0 & 0.0 & 0.0 & 0.0 \\
        -6.0 & 0.5 & -4.0 & 0.5 \\
        4.0 & -7.0 & 3.5 & -0.5 \\
        -1.0 & 1.833 & -1.0 & 0.166
    \end{bmatrix} \\
\end{align*}
\begin{align*}
        B_i(X) \\
    \begin{bmatrix}
        3.0 & -5.166 & 2.5 & -0.333 \\
        -2.0 & 5.166 & -2.5 & 0.333 \\
        0.0 & 0.0 & 0.0 & 0.0 \\
        0.0 & 0.0 & 0.0 & 0.0 \\
        0.0 & 0.0 & 0.0 & 0.0 \\
        0.0 & 0.0 & 0.0 & 0.0
    \end{bmatrix}
\end{align*}
\begin{align*}
        C_i(X) \\
    \begin{bmatrix}
        0.0 & 0.0 & 0.0 & 0.0 \\
        0.0 & 0.0 & 0.0 & 0.0 \\
        -1.0 & 1.833 & -1.0 & 0.166 \\
        4.0 & -4.833 & 1.5 & -0.166 \\
        -6.0 & 9.5 & -4.0 & 0.5 \\
        4.0 & -7.0 & 3.5 & -0.5
    \end{bmatrix}
\end{align*}
The corresponding values in the matrices represent polynomial coefficients and shall be read from right to left, e.g. \(A1(X) = 0.833x^3 - 5x^2 + 9.166x -5\). For example:
\begin{align}
     X = 1 \\
    A1(1) = 0, A2(1) = 1, A3(1) = 0, A4(1) = 0, A5(1) = 0, A6(1) = 0
\end{align}
Comparing the results with the first vector a of the first gate, we see that the results represent the first gate.
For example, evaluating \(A, B \text{ and }C\) at X = 1 means to adding up the coefficients of the first polynomial of \(A\) , which will result in a value matching to the vector value in the first gate. Then, the next polynomial in A etc. It will result in a vector of length six and will be correct if the values match vector a from the first gate in our R1CS. The same is done by evaluating the polynomials with X, whereby X starts at 1 and ends at the number of gates, in our case \(X=\{1,2,3,4\}\).
However, it would be cumbersome to evaluate each constraint individually. This is why we can make use of the QAP to check whether the dot product equation of the polynomials will hold:
\begin{align}
    A_i(X)\cdot \Vec{s} * B_i(X)\cdot \Vec{s} - C_i(X)\cdot \Vec{s} = H(X) * Z(X)
\end{align}
Interestingly, the left side of the equation is our target polynomial \(T(X)\), which we want to proof.
Now, let's have a look at the right side of the equation in (6.6). \(Z(X)\) is known if we know the number of constraints. In Groth16, it is made available in the trusted setup. In this case, we have four gates, so we arrive at
\begin{align}
    Z(X) = (x-1)(x-2)(x-3)(x-4)
\end{align}
\(H(X)\) is the hiding of our initial minimal example, those input we don't want to share, but prove we know the solution to. What role hiding plays will be explained shortly. Now, we want to still finish looking at the equation in (6.6). In essence, we want to proof we know a polynomial and its solution, so that
\begin{align}
    T(X) / Z(X) = H(X)
\end{align}
Ultimately, in our example, \(H(X)\) is also a polynomial. We know the QAP is correct, if \(H(X)\) is a polynomial without remainder. In this example, the resulting
\[H(X) = -0.44x^3 + 17.055x^2 - 3.666x.\]

Practically, the coefficients of each polynomial in \(A_i(X), B_i(X) \text{ and }C_i(X)\) are publicly known. The same can be said for \(Z(X)\) through knowing the number of constraints (in this example we have four constraints). The prover can calculate the coefficients of \(H(X)\) by dividing T(X) / \(Z(X)\). However, there is no zero-knowledge yet, since the prover has to prove knowledge of vector \(s\) and \(H(X)\) without revealing it.
\newpage

\subsubsection{Hiding}
Zk-SNARKs are dependent on a trusted setup releasing these parameters. The goal is to prove knowledge of the polynomial \(H(X)\) with all its coefficients without disclosing any of this information. Therefore, the trusted setup also provides a random secret point \(P\). Note, that \(P\) is hashed, calculated once and deleted from memory. Depending on the number of constraints, a certain amount of \(P\) values are needed. In our example, we have four constraints, which need \(P = {1, P, P^2, P^3}\), whereby the value of \(P^3\) corresponds to the value of \(x^3\) when the polynomials are evaluated. The following values of \(P\) are provided

\begin{align}
    hh(1), hh(P), hh(P^2), ..., hh(P^\textsuperscript{(no. of constraints - 1)})
\end{align}

The trusted setup makes these values publicly available in the CRS.

With our previous knowledge, we know that the prove will consist of
\begin{align}
    \frac{hh[A(P)] * hh[B(P)] - hh[C(P)])}{hh[Z(P)]} = hh[H(P)]
\end{align}

The hidings of our polynomials are numbers, that currently can just be forged. The following will show how it can be proven that these numbers are hidings of the polynomials \(A(X)\), \(B(X)\) and \(C(X)\) in \(P\) which is not known to anybody. Furthermore, we need to prove that in order to arrive at \(A(X)\), \(B(X)\) and \(C(X)\), the same solution vector \(s\) was used (6.6). 

In order to approach the first problem, proving that the hidings of \(A(X)\),  \(B(X)\) and \(C(X)\) were actually calculated in \(P\), we need to "extend" \(P\) by the same number, namely \(u\). The CRS consists also of \(hh(u*P), hh(u*P^2), hh(u*P^3)\), etc., i.e., it consists of two sets of hidings. We know that \(A(X)\) is a linear combination of the values of vector s inserted into the polynomials A1, A2, A3, etc. of A. B calculating \(hh[A(P)]\) and \(hh[A(u*P)]\) and looking if \(hh[A(P)] = u * hh[A(u*P]\) holds true, shows that the hiding of \(A(X)\) calculated in \(P\)is indeed a result of linear combination of A1, A2, A3, etc. and the values of the vector s (6.6). All we did is to prove that the same sets of hidings of \(P\)were use to arrive at these numbers. The same is applied to the hidings of \(B(X)\), \(C(X)\) in P.

For the second problem, to prove the same values of vector s were used to arrive at the hidings of \(A(X)\), \(B(X)\) and C(X) in \(P\) a similar approach can be used. In our example, vector s has six solution variables. We use a new variable K as
\begin{align}
    K = K1 + K2 + K3 + K4 + K5 + K6
\end{align}
\begin{align*}
    K1 = A1(P) + B1(P) + C1(P)\\K2 = A2(P) + B2(P) + C2(P)
\end{align*}
\begin{center}
    ... \\
\end{center}
\begin{align*}
    K6 = A6(P) + B6(P) + C6(P)
\end{align*}
By checking that 
\begin{align}
    hh[K(P)] = one*hh[K1] + x * hh[K2] + out * hh[K3] + ... + sum2 * hh[K6],
\end{align}
we can prove that indeed the same coefficients of vector s were used. This way it is nearly impossible to come up with numbers that hold true for another \(P\)and to create proofs without the knowledge of the coefficients.

\subsubsection{Homomorphic Hiding}

\(y = hh(x)\) is a hashing function. It is collision resistant, i.e., one cannot guess anything of x from y. For zero-knowledge proofs, this property alone is not sufficient. The hashing function should also preserve algebraic structures, so the checks in , e.g., (6.10) can be performed. Let us divide the term \textit{Homomorphic Hiding} into two sections to explain in more detail.

A function \(y = hh(x) = e^x\) is homomorphic if
\begin{align}
    hh(a*x1 + b*x2) = e^\textsuperscript{a*x1+b*x2} = e^\textsuperscript{a*x1} * e^\textsuperscript{b*x2} = hh(x1)^a * hh(x2)^b
\end{align}
As seen in (6.13), the basic exponential laws hold. However, this function is not hiding, because one could calculate the logarithmic base e of x, because of working with only real numbers \begin{math}\mathbb{R}  
\end{math} so far.

We need to express variables in a finite field as of modulo p with p being a large prime. The finite field consists only of integer inputs in the range of 1 and some value p-2. This way, expressing values in modular arithmetic, nobody can guess or calculate our base e anymore. Now,
\begin{align}
    y = hh(x) = G^x,
\end{align}
where G is a value in the finite field \begin{math}\mathbb{F}_p\end{math} and y will always be expressed as modulo p.

With homomorphic hiding being introduced, we know all the tools being used to prove that we can calculate the equation in (6.8) with the polynomials from the QAP and the same values of vector s, without knowing any \(P\)and u * P. We know how the proof is calculated without revealing our solution vector s. The following deals with the verification if the above equations hold true, without revealing the solution vector s.

\subsubsection{Elliptic Curve Pairing}

The goal of PCP and pairing-based zero-knowledge algorithms is to create a succinct proof, that a defined computation with given inputs produces certain known outputs, without revealing any information about them and to show that the constraints of that computation hold. Eventually, we want to check if the following equality holds true, i.e., that after transforming our problem into polynomial structure, we know some polynomials so that

\begin{align}
    \frac{A(x) * B(x)}{Z(x)} = H(x) + C(x)
\end{align}

We have the polynomials \(A, B \text{ and }C\), not expressed in real numbers, but mapped to a finite field with a large prime number. We can calculate \(H(X)\) as in (6.8). Now, we are going to use generators for each of our polynomial to produce points on an elliptic curve. This is necessary to make use of pairing, which allows us to check if equations, e.g., (6.13), hold true without knowing the actual variable values in these equations. In the following, some preliminaries will be introduced to create a basis for the Groth16 CRS generation, and proofing and verification mechanism.

Elliptic curves are used to define collision resistant one-way functions, i.e., homomorphic hiding functions. An elliptic curve is a polynomial, e.g., the elliptic curve used in Bitcoin (Figure 1). 

\begin{figure}
\centering
\begin{minipage}{.5\textwidth}
  \centering
  \includegraphics[width=.4\linewidth]{Pictures/bitcoinec.png}
  \caption{Figure 1:\ \(y^2 = x^3 +7\)}
  \label{fig:test1}
\end{minipage}%
\begin{minipage}{.5\textwidth}
  \centering
  \includegraphics[width=.4\linewidth]{Pictures/y2x3.png}
  \caption{Figure 2:\ \(y^2 = x^3\)}
  \label{fig:test2}
\end{minipage}
\end{figure}

Elliptic curves are useful for zk-SNARKs because of the discrete logarithm problem, which is believed to be very hard to solve. Given a point \(g\) on the elliptic curve, and a multiple of that point, \(n*g\), it is impossible to solve n, even if \(g\) and \(n*g\) are given. In order to choose an elliptic curve that offers homomorphic hiding, we need to implement a mapping between our known numbers of the finite field \begin{math}\mathbb{F}_p\end{math} and a set of points on the elliptic curve (hidden space). Let \begin{math}\mathbb{F}_p\end{math} be a finite field of order p, whereby p is a large prime, e.g., if \(p=97\), then \begin{math}\mathbb{F}_p\end{math} \(=\{0, 1, 2, 3, ..., 96\}\) For this, we are going to take a generator point \(g = (x1,y1)\) that lies on the elliptic curve and multiply it with every element \({element}_i\) in \begin{math}\mathbb{F}_p\end{math}. For example, \(g + g = 2*g\) is calculated by putting a tangent line on \(g\), and wherever the line crosses the elliptic curve, we receive the result by using the opposite signs of that point. To arrive at \(2*g + g = 3*g\), the point \(2*g\) is used to draw a line to \(g\), see where the line further intersects with the elliptic curve, and use the opposite signs of that point to arrive at \(3*g\). This is repeated for every \(element\) in \begin{math}\mathbb{F}_p\end{math}. As a result, we have our finite field mapped to a hidden space on the elliptic curve. In summary, every \(element\) is hidden by
\begin{align}
    hh(element) = element * g
\end{align}
Additionally, we have to define what 0 and 1 are. The element 0 is the subtraction of a point on the elliptic curve, i.e., when point g goes to infinite. 1 is the point g itself.
Now, we have achieved homomorphic addition:
\begin{align}
    (A + B) \longrightarrow (A + B) * g = A*g + B*g
\end{align}

In order to use elliptic curve pairing to verify zk-SNARKs proofs, e.g., Groth16, we need to achieve a limited homomorphic multiplication operator. The hidden space is a group of points generated by the finite field elements and the generation point, \(g1\) on the elliptic curve. Now, we want to choose a subgroup \begin{math} \mathbb{G}_1\end{math} from that group. We choose that subgroup in a away that the number of elements we chose, \(r\), is a prime number too. Having found r, we can continue to choose the embedding degree of the elliptic curve. In Groth16, the proving key and verification key consist of \begin{math} \mathbb{G}_1\end{math} and \begin{math} \mathbb{G}_2\end{math} element. The embedding degree \(k\) has to be found in a way that \(p^k-1\ |\ r\), i.e., is a multiple of. Let us use a minimal example to show how to arrive at \begin{math} \mathbb{G}_1\end{math} and \begin{math} \mathbb{G}_2\end{math}.
Let us define an example base field \begin{math}\mathbb{F}_p\end{math} \(= \{0,1,2\}\) with \(p = 3\). We have found an embedding degree \(k=2\). In order to achieve our goal of creating a subgroup \begin{math} \mathbb{G}_2\end{math}, we need to extend our base field by a defining polynomial. This polynomial is of degree \(k\), and no element of our base field evaluates it to 0. In summary, we have:

\begin{align}
    \mathbb{F}_p = \{0,1,2\}, p = 3, k = 2
\end{align}

Defining polynomial for field extension \begin{math}\mathbb{F}_p^k\end{math}: 
\begin{align*}
    f(x) = z^2 +1 \\
    f(0) = 1\\
    f(1) = 2\\
    f(2) = 5 mod 3 = 2
\end{align*}

As shown in (6.18), none of the base field elements make f(x) evaluate to 0. In order to create the elements of the field extension \begin{math}\mathbb{F}_p^k\end{math}, we have to create all possible degree 2 polynomials out of the combinations of our base field \begin{math}\mathbb{F}_p\end{math}. For example, one possible polynomial with the coefficients from our base field is:

\begin{align}
    1*z^2+2*z+0 
\end{align}   
\begin{align*}
    (z^2+2*z)\mod (z^2+1) = 2*z-1\mod 3 = 2*z+2
\end{align*}
\(2*z+2\) is one element of the extension field \begin{math}\mathbb{F}_p^k\end{math}. In total, \begin{math}\mathbb{F}_p^k\end{math} has 9 elements, all calculated as in (6.18). In Summary, the elements of our extension field \begin{math}\mathbb{F}_p^k\end{math} are
\begin{align}
\{0, 1, 2, z, z+1, z+2, 2z, 2z+1, 2z+2\}
\end{align}

As shown in (6.20), the elements of the extension field are polynomials of degree up to \(k-1\). Addition and multiplication are defined in the way that coefficients are calculated \(\mod 3\) and polynomials \(\mod z^2+1\), the defining polynomial  
\(f(x)\) from (6.18).

Now, having our extension field, we can use it to create \begin{math}\mathbb{G}_2\end{math}, a subgroup of points of the same elliptic curve used for \begin{math}\mathbb{G}_1\end{math}, but with elements of \begin{math}\mathbb{F}_p^k\end{math}, instead of base field \begin{math}\mathbb{F}_p\end{math}. For this, we have to define points, whereby x and y coordinates are polynomials from \begin{math}\mathbb{F}_p^k\end{math}. \begin{math}\mathbb{G}_2\end{math} will consist of combinations from \begin{math}\mathbb{F}_p^k\end{math} in the form of \((x,y)\), which satisfy the elliptic curve. 

Pairings are bilinear maps that combine elements of two spaces to receive an element of a third space, e.g., matrix multiplication. In Groth16, the following pairing notation is used:

\begin{align}
    e: \mathbb{G}_1 \times \mathbb{G}_2 \to \mathbb{G}_T
\end{align}

The result of all steps performed previously is an incomplete homomorphic multiplication that enables us to check that the correct polynomial coefficients were used for \(A(x), B(X), \text{ and }C(x)\), as well as the same solution vector \(s\). It is incomplete, because not more than two elements can be multiplied. However, this just satisfies the use case for zk-SNARKs. 

\subsubsection{Groth16}

Preliminaries to understand the Groth16 protocol have been covered. In the following, we will describe the setup, proof, and verification steps in Groth16. The parameters are summarized in Table \ref{tab:Groth16Params}.

\setlength{\tabcolsep}{2ex}
\renewcommand{\arraystretch}{1.5}%
\begin{table}[hbt]
	\centering
	    \caption{Groth16 given parameters}
		\begin{tabular}{| m{0.35\linewidth} | m{0.6\linewidth} |}
		\hline
		\textbf{Parameter} & \textbf{Definition}\\ \hline
            \(n,m\) & number of constraints, number of variables\\ \hline
            \begin{math}\mathbb{F}_p\end{math} & finite field of prime order p\\ \hline 
            \begin{math}\mathbb{G}_1, \mathbb{G}_2, \mathbb{G}_T\end{math} & groups of points of prime order p satisfying an elliptic curve\\ \hline
            \begin{math}\mathbb{G}_1 \times \mathbb{G}_2 \to \mathbb{G}_T\end{math}& bilinear pairing \\ \hline
            \begin{math}g_T = e(g_1, g_2)\end{math}& generators with mapping \\\hline
            \begin{math}\bigl\{A_i(X), B_i(X), C_i(X)\bigl\}_{i=0}^m\end{math} & encoded computation as result of R1CS and QAP three sets of polynomials of degree \(n-1\)\\ \hline
            \(Z(x) = (x-1)*(x-2) * \newline (x-3)...(x-(n-1))\) &  minimal polynomial, known because n is known \\ \hline
            \(l\) & number of public inputs \\ \hline
            \((s_1,...,s_l)\) & elements of witness whose inputs are public \newline (e.g., out = 35 in our example) \\ \hline
            \((s_{l+1},s_{l+2},...s_m)\) & elements of witness for secret input x, with \(s_0 = 1\) \\ \hline
	\end{tabular}
\label{tab:Groth16Params}
\end{table}

\subsubsection{Key generation}

The proving and verification key are obtained from the Common Reference String (CRS) via multi-party computation. From \begin{math}\mathbb{F}_p\end{math}, a set of random values is generated. This toxic waste (tw), or trapdoor, must be secret and forgotten from memory, because knowledge of it enables forged proofs. Note, that \begin{math}\tau\end{math} is the random point \(P\) from our examples. From the toxic waste, polynomial \(L_i(x)\) is defined:
\begin{align}
    tw = (\alpha, \beta, \gamma, \delta, \tau) 
\end{align}
\begin{align*}
    L_i(x) = \beta * A_i(X) + \alpha * B_i(X) + C_i(X)
\end{align*}
The CRS consist of \begin{math} \sigma = ([\sigma_1]_1,[\sigma_2]_2)\end{math}, which are elements of \begin{math} \mathbb{G}_1, \mathbb{G}_2\end{math}.

\begin{align}
    [\sigma_1]_1 = 
    &\ [(\alpha, \beta, \gamma, \delta, \\
    &\ 1, \tau, \tau^2, \tau^3, ..., \tau^{n-1}, \\
    &\ \frac{L_0(\tau)}{\gamma}, ..., \frac{L_l(\tau)}{\gamma}, \\
    &\ \frac{L_{l+1}(\tau)}{\delta}, ..., \frac{L_m(\tau)}{\delta})]_1 
\end{align}
\begin{align*}
    [\sigma_2]_2 = [(\beta, \gamma, \delta, \ 1, \tau, \tau^2, \tau^3, ..., \tau^{n-1})]_2
\end{align*}

\begin{itemize}
    \item (6.23): elements of the toxic waste
    \item (6.24): powers of \begin{math}\tau\end{math} of degree up to \(n-1\)
    \item (6.25): the polynomial is chosen from the set of polynomials of \(A(X), B(X), C(X)\), which corresponds to the place of the public input of the solution vector. In our starting example, \(out = 35\) is the public input (since this is our only public input, \(l=1\)). The public input is at third place in s. Hence, \(A_3(X), B_3(X), C_3(X)\) are chosen, evaluated at \begin{math}\tau\end{math} and multiplied by \begin{math} \alpha, \beta\end{math} according to (24).
    \item (6.26): Same as (6.25), but for the non-public inputs of s. All elements of \begin{math} [\sigma_1]_1 \end{math} are \begin{math}\mathbb{G}_1\end{math} elements, e.g., \begin{math}\alpha_1 = g_1 * \alpha\end{math}.
\end{itemize}

The proving key consists of the following elements:
\begin{itemize}
    \item \([(\alpha, \beta, 1, \tau, \tau^2, \tau^3, ..., \tau^{n-1}, \frac{L_{l+1}(\tau)}{\delta}, ..., \frac{L_m(\tau)}{\delta})]_1\)
    \item \([(1, \gamma, \delta)]_2\)
    \item circuit information about the polynomials: \\
    \(A_0(X), A_1(X), ..., A_m(X)\),\\
    \(B_0(X), B_1(X), ..., B_m(X)\),\\
    \(C_0(X), C_1(X), ..., C_m(X)\),\\
    \(Z(x) = (x-1)(x-2)(x-3)...(x-(n-1))\)\\
\end{itemize}

The verification key consists of the following elements:
\begin{itemize}
    \item \([(1, \frac{L_0(\tau)}{\gamma}, ..., \frac{L_l(\tau)}{\gamma})]_1\)
    \item \([(1, \gamma, \delta)]_2\)
    \item precomputed pairing \([\alpha * \beta]_T\), which is a \begin{math}\mathbb{G}_T\end{math} element
\end{itemize}

\subsubsection{Generating the proof}

Two random numbers \(r, t\) are generated from \begin{math}\mathbb{F}_p\end{math}, that are used to compute

\begin{enumerate}
    \item \begin{math} A= \alpha + s_0*A_0(\tau) + s_1*A_1(\tau) + ... + s_m*A_m(\tau) + r\delta\end{math}
    \item \begin{math} B= \beta + s_0*B_0(\tau) + s_1*B_1(\tau) + ... + s_m*B_m(\tau) + t\delta\end{math}
    \item \begin{math} C= \frac{s_{l+1}L_{l+1}(\tau)}{\delta} + \frac{s_{l+2}L_{l+2}(\tau)} + ... +\frac{s_{lm}L_{lm}(\tau)}{\delta} + \frac{H(\tau)Z(\tau)}{\delta} + At + Br - rt\delta\end{math}
\end{enumerate}

The proof \begin{math}\pi\end{math} consists of two elements from \begin{math}\mathbb{G}_1\end{math} and one element from \begin{math}\mathbb{G}_2\end{math}:
\begin{align}
    \pi = ([A]_1, [B]_2, [C]_1)
\end{align}

\subsubsection{Verification}

In Groth16, three pairings are checked during verification. \begin{math}[\alpha * \beta]_T\end{math} is a precomputed pairing and is made available in the setup phase. The verification computation receives proof \begin{math} \pi\end{math} and accepts it only if the following equation holds:
\begin{align}
    [A]_1 * [B]_2 = [\alpha]_1[\beta]_2 + \bigl[\frac{s_0L_0(\tau)}{\gamma}+ \frac{s_1L_1(\tau)}{\gamma} + ... + \frac{s_lL_l(\tau)}{\gamma}\bigr]_1 * [\gamma]_2 + [C]_1 * [\delta]_2
\end{align}

As shown in (6.27), the following three pairings are needed to be checked:

\begin{itemize}
    \item \(e([A]_1, [B]_2)\)
    \item \begin{math}
        e(\bigl[\frac{s_0L_0(\tau)}{\gamma}+ \frac{s_1L_1(\tau)}{\gamma} + ... + \frac{s_lL_l(\tau)}{\gamma}\bigr]_1 , [\gamma]_2)
    \end{math}
    \item \begin{math}
        e([C]_1, [\delta]_2)
    \end{math}
\end{itemize}

Let us evaluate the verification equation in (6.27). The left hand side evaluates as follows:

\begin{equation*}
\begin{split}
    [A]_1 * [B]_2 = [A*B]_T &= [\alpha + s_0*A_0(\tau) + s_1*A_1(\tau) + ... + s_m*A_m(\tau) + r\delta]_1 \ *\\
    &\ \ \ \ [\beta + s_0*B_0(\tau) + s_1*B_1(\tau) + ... + s_m*B_m(\tau) + t\delta]_2 \\
    &= [(\alpha + A(\tau) + r\delta) * (\beta + B(\tau) + t\delta)]_T\\
    &= [\alpha * \beta]_T + [\alpha * B(\tau)]_T + [\alpha * t\delta]_T \ + [A(\tau) * \beta]_T \ + \\
    &\ \ \ \ [A(\tau) * B(\tau)]_T + [A(\tau) * t\delta]_T + [r\delta * \beta]_T + [r\delta * B(\tau)]_T +\\
    &\ \ \ \ [r\delta * t\delta]_T \\
    \\
    &= [A(\tau) * B(\tau)]_T \textcolor{blue}{\ +\ [\alpha * \beta]_T + [\alpha * B(\tau)]_T + [\alpha * t\delta]_T} \\
    &\ \ \textcolor{blue}{+ [A(\tau) * \beta]_T \ + [A(\tau) * t\delta]_T + [r\delta * \beta]_T + [r\delta * B(\tau)]_T} \\
    &\ \ \textcolor{blue}{+ [r\delta * t\delta]_T}
\end{split}
\end{equation*}

The right hand side evaluates to:
 \begin{equation*}
     \begin{split}
    &=[\alpha]_1[\beta]_2 + \bigl[\frac{s_0L_0(\tau)}{\gamma}+ \frac{s_1L_1(\tau)}{\gamma} + ... + \frac{s_lL_l(\tau)}{\gamma}\bigr]_1 * [\gamma]_2 + [C]_1 * [\delta]_2 \\
    &=[\alpha * \beta]_T + [(s_0L_0(\tau) + s_1L_1(\tau) + ... + s_lL_l(\tau))]_T + [(s_{l+1}L_{l+1}(\tau) + s_{l+2}L_{l+2}(\tau) + ... \\
    &\ \ \ + s_{lm}L_{lm}(\tau)) + H(\tau)Z(\tau) + At\delta + Br\delta - rt\delta\delta]_T
     \end{split}
 \end{equation*}

Now, we can replace A and B. We also see that the middle of the equation is \(L_i(\tau)\).
 \begin{equation*}
     \begin{split}
     &=[H(\tau) * Z(\tau) + C(\tau)]_T \textcolor{blue}{\ +\  [\alpha * \beta]_T + [\alpha * B(\tau)]_T + [\alpha * t\delta]_T + [A(\tau) * \beta]_T} \\
     &\ \ \ \textcolor{blue}{+ [A(\tau) * t\delta]_T + [r\delta * \beta]_T + [r\delta * B(\tau)]_T + [r\delta * t\delta]_T}
     \end{split}
 \end{equation*}

Eventually, we get the equality check we wanted to achieve (6.15). The use of the secret encoded values \(\alpha, \beta\) of the toxic waste (6.22) force the prover algorithm to use the same coefficients of the solution vector (witness) to compute \(A(X), B(X), C(X)\). \(\gamma, \delta\) ensure that the public inputs of order \(l\) are independent from the solution vector (witness). In order to achieve the zero-knowledge aspect, \(r, s\) are used to randomly shift the proof.

\section{Plonk-based zk-DApp}
The decentralized application for zero-knowledge MRO data attestation (zk-DApp) facilitates data entries and data sharing between MRO data owners (submitters) and aviation authorities (attesters) in zero-knowledge. This software artifact attempts to ensure compatibility between the lack of trust among industry participants and the need for data verification and transparency (trade-off between requirements within the project).
\subsubsection{Circom, snarkjs}
The zero-knowledge part of the zk-DApp is implemented via circom and snarkjs, which are developed by iden3. iden3 is an open-source project focused on scalable and distributed identity systems on zero-knowledge proofs, i.e., providing protocols, data structures, and modules. The current objectives include achieving self-sovereign identities free of charge, minimizing on-chain transactions, re-designing privacy with non-reusable proofs, creating complete products, e.g., user wallets, and focusing on community standardization \citep{iden3aboutus}. Circom is Rust-based and allows to write and compile arithmetic circuits. In circom, there is a data type called field elements, which are values modulo a large prime number and based on the BN128 elliptic curve by default \citep{circom} By compiling the circuit, R1CS files are generated: the constraint system in binary format, a symbols file for debugging, and files that are necessary to generate the witness. In order to compute the witness, an input file in json format needs to be specified, that contains the witness information. In the zk-DApp, the submitter enters the information to be verified, which is processed according to an input schema. The witness is computed and saved in a file format accepted by snarkjs. Snarkjs is a JavaScript library that continues with the outputs from circom and executes a specified zero-knowledge proof protocol, e.g., currently, Groth16, Plonk, and Fflonk are supported \citep{snarkjsdoc}. For this use case, snarkjs' plonk implementation is utilized. For the verification, as well as intermediate result hashes, a powers of tau file is utilized, so that the multi-party computation can be performed. In Groth16, it is utilized during the trusted setup (5.2.1). In plonk, this is the source of public randomness (5.2.2). Figure \ref{fig:zk-DAppgeneral} shows the main steps performed for a plonk-based zk-DApp. Plonk does not require a trusted ceremony for each circuit, i.e., the powers of tau ceremony is universal. After compiling the circuit and calculating the witness, the snarkjs plonk can be setup. A final protocol transcript key (zkey) is generated and verified. It is the zero-knowledge key that includes proving and verification key \citep{snarkjsdoc}. From the zkey, the verification key is exported in json format. With this verification key file, the final zkey file and the witness file, the proof file, and the file with the public inputs and outputs are generated. These two files and the verification key are used to verify the proof. The verifier is transformed into a smart contract (PlonkVerifier.sol), that is performing the verification. In order to utilize the PlonkVerifier smart contract that is exported in the previous step, an interface needs to be created (IPlonkVerifier.sol), which initiates verifyProof and outputs a true or false. The ZkDocument.sol smart contract, which stores user inputs, e.g., the commitments and trusted attesters, uses this interface to verify the proof and update the states of the smark contracts. With every user input, a witness and proof are calculated. The commitment is posted and the smart contracts updated. Once verified, the user interface (UI) shows an update, e.g., a green highlighting. 
\begin{figure}[hbt]
	\centering
		\includegraphics[width=1.0\textwidth]{Pictures/circom snarkjs process flow.png}
	\caption{plonk-based zk-DApp high-level process flow}
	\label{fig:zk-DAppgeneral}
\end{figure}
For deployment, broadcasting, and development, hardhat testnet is used. Metamask is used as crypto wallet browser plugin.

\subsection{Implementation and Results}
In the following, the different roles and the steps to be performed in the zk-DApp are described. First, the process flow and general functionalities are depicted. Second, important implementation logic and features are highlighted with references to the implementation code. Lastly, open items and limitations are discussed.

For the zk-DApp to start, the command \textit{yarn start} needs to be executed in the zkdocs-backend directory. It will start the hardhat test network and show a set of test accounts with private keys to use. These test accounts are also used for the accounts of the verifier, submitter and attesters. The next command in Figure \ref{fig:second-cmd} is also run in the same directory and deploys the schema for the MRO data entry. It triggers the circom and snarkjs ceremonies and shows the number of constraints, inputs, outputs, wires, and labels in the arithmetic circuit. Also, it shows that the corresponding R1CS, symbols and witness generating files are written successfully. Once, the setup is finished, the public keays of the deployer (submitter), plonk verifier (PlonkVerifier.sol smart contract address), and zkDoc (ZkDocument.sol smart contract address) are shown. The zkDoc public key needs to be parsed every time another set of inputs needs to be made, a verification needs to be performed, or to see the latest verification. In the end, the schema is created and the corresponding directories are listed. The UI can be started with the same command \textit{yarn start} in the zkdocs-ui directory.

\begin{figure}[hbt]
	\centering
		\includegraphics[width=1.0\textwidth]{Pictures/second-cmd.png}
	\caption{Initiating the deployment and plonk in circom, snarkjs in the zk-DApp backend}
	\label{fig:second-cmd}
\end{figure}

\subsubsection{Sequence of events}
In the zk-DApp, there are three roles: verifier, submitter, and attester. In this scenario, the verifier also acts as administrator, operating the zk-DApp. Hence, the administrator operates the schema with constraints, fields and the list of trusted institutions that are capable to attest in the zk-DApp. The submitter is in possession of MRO data, and seeks validation. The attester is assigned the task to view specific information shared with them and can validate it, i.e., attest to the data provided. The verifier can obtain the proof and verify it. Other stakeholders in the system can learn, that there is verified information, but do not learn anything besides its correctness. The following sequence of events refers to the landing page of the zk-DApp (Figure \ref{fig:landing-page}).
\begin{figure}[hbt]
	\centering
		\includegraphics[width=1.0\textwidth]{Pictures/landingpage.png}
	\caption{zk-DApp landing page}
	\label{fig:landing-page}
\end{figure}
\begin{enumerate}
\item \textit{Fill Form}:
The submitter connects their wallet and can see an input mask for MRO data entries (Figure \ref{fig:form}. In this first prototype, the submitter can enter the part identifier, flight hours, time since last service, and MRO cycles. Once the values are added, the submitter decides which authority needs to attest to the values, e.g., FAA/Unites States, EASA. If the constraints are met, the submitter generates the proof, which results in the following: first, the proof is generated and displayed. Second, there are communication links generated, that can be parsed in another browser tab. These links are specific to the respective attesters, so the data assigned to them can be viewed. In practice, this has to be realized via separate communication channels to alert attesters, once a new set of MRO data inputs was performed. The submitter commits to these entries and the proof on-chain. Now, the values can be viewed by the respective attester and validated.

\item \textit{Attest to MRO data}: The respective attester follows the attestation link, connects their wallet, and only sees the decrypted values of the data assigned to them for validation. The attester marks the data as true and commits the result. Once all attesters have successfully attested to the data, the verification can be initiated.

\item \textit{Verify Form}: Another user, e.g., the administrator, the submitter, or any dedicated user with the proof at hand, can perform the verification. The verifier connects their wallet and checks the attestations. The only information available is the green mark, that data has been successfully attested to, which fields have been attested to, and who attested to these fields and values. The values are shown in encrypted format. With the proof parsed, the verifier can verify and submit. If the attestations are correct, a bit will be flipped in the smart contract \citep{zkdocs}, and the verification is successfully performed.

\item \textit{View Verified}: Any participant can view the latest verified information. However, only the green mark is shown of whether the data is correct, the field names, and which trusted authority performed the attestation. The verifier and other participants do not learn anything besides that the information provided is correct and has been truthfully attested by a specific trusted institution.
\end{enumerate}

\begin{figure}[hbt]
	\centering
		\includegraphics[width=1.0\textwidth]{Pictures/form.png}
	\caption{zk-DApp Fill Form (first implementation)}
	\label{fig:form}
\end{figure}

\subsubsection{Highlighted features}
The following features are described in more detail to illustrate the underlying functionality applied in the zk-DApp.

\begin{itemize}

\item The input for the circuit and witness calculation is processed through a schema. The json template can be seen as rule book for the input field and values, constraints, and trusted attesting addresses. \textit{ZkDocSchema.ts} uses the json template to define the schema. It sets valid operators, constraint types, and validates if the template matches to those. Some further value validations are build-in as well, e.g., checks for non-negative numbers. 

\item \textit{ZkDocGenerator.ts} acts as wrapper for circom and snarkjs. It contains all necessary commands to initiate the circuit, represent intermediate formats, e.g., R1CS, calculate the witness and the zero-knowledge key file, as well as exporting the verifier as smart contract. It generates the constraint string and looks up the field indices of the fields to attest in the schema. The circuit with its functionality is specified in \textit{circuit.circom}. It defines the constraints checks for each field, if the commitment matches the hashed commitment on-chain. During attestation, the assigned attester performs the decryption of the values locally.

\item The actual user input values are handled in the \textit{ZkDocument.sol} smart contract. It stores the selected attester addresses, the submitter's address, field commitments, and field indices to be attested to. For each field and value, a nonce is generated and the commitment scheme is initiated. The commitment scheme for each field is \textit{hash(value, nonce)}. The hashing algorithm used in the circuit is Poseidon. Due to its reduction of prover and verifier complexity, Poseidon is currently favored in research and development of zero-knowledge decentralized applications. Unlike other popular hashing functions, e.g., SHA-256, Poseidon is able to operate on smaller circuits and is adapted to finite fields \citep{poseidon}. It can add and remove trusted attesters from the list via \textit{(addValidInstitution, addValidInstitutions, removeValidInstitution)}. It makes sure, that only the attester specified by the submitter via mask entry is able to attest to the fields and values committed. During this check, it is also ensured that the field index of a attestation matches with the one specified during submission. Once the check is successfull, the bit is flipped to confirm correctness of the attestation. This procedure is applied to all field attestations at once via \textit{attestMultiple}. The function \textit{postFields} posts the commitments and checks that 1) the commitment length matches the number of fields, 2) that the correct number of institutions to perform attestations are in the commitment, and 3) that the commitment is unique and has not been posted previously. Through \textit{validateSubmitter} the public signal array for plonk verification is filled with the commitments. It checks that all fields have been attested to by the specified institutions. As described in the introduction of circom and snarkjs in 6.2, the verifier is transformed to a Solidity smart contract \textit{PlonkVerifier.sol}), containing the values and function logic needed for plonk verification. \textit{ZkDocument.sol} uses the interface to \textit{PlonkVerifier.sol}, the \textit{IPlonkVerifier.sol} to initiate the function \textit{verifyProof}, which verifies the proof. By using the interface, it is made sure that a specific standardized function from the generated \textit{PlonkVerifier.sol} is used (\textit{verifyProof}). This way of implementation makes sure, that the state of \textit{PlonkVerifier.sol} undergoes an actual change. In order to prevent client-side attacks, the schema hash, using the Keccak-256 hashing algorithm (part of SHA-3 Solidity family) is stored in \textit{ZkDocument.sol} as well. 

\item The UI finds the correct schema via the scheme hash stored, which is facilitated through \textit{SchemaUtils.ts}. From the schema hash, it obtains the corresponding schema and can utilize the files generated through the plonk procedure in circom and snarkjs (6.2).
\end{itemize}
\begin{comment}
ÜBERLEITUNG: 
- use case support system for predictive maintenance
- MRO KPIs are found in different documents, have to be attested by different institutions: e.g. shop report -->repair shop, release certificate-->aviation authority
- also for trading this is important information 
- this use case resulted from requirement privacy and transparency as well as trading of parts
- assumption: data is available
- überleitung zur daten architektur: automatic verification of part id and its true existence through merkle tree zk verification
\end{comment}

\section{Zero-Knowledge Data Structure}
- USE CASE: check the authenticity of MRO spare part documentation-->currently hard to achieve due to paper based documentation
- big need to prove authenticity, e.g., through checking if mechanic and date stempel match with a part if and certificate id
- big need to strive towards a decentralized temper-proof data structure
-create a architecture for verification mechanisms in Rapado, based on Sedlmeier, Völter, Strüker (2021):
-->proof membership in a merkle tree? data structure of parts, certificates

- Campanelli et al 2022: improvement on zkSNARKS with Merkle Trees -->HARISA, could be used for the architecture

- build zk DAPP for document verification with circom snarkjs on hardhat test network
- maybe use snarkjs Groth16 first and then PLONK as future outlook, when TurboPLONK support will come in snarkjs

\section{Evaluation}

\subsubsection{zk-DApp MRO Attestations}
On the developer's machine with 2,2 GHz 6-Core Intel Core i7 processor and macOS 13.2.1 (22D68) operating system, the circom build, snarkjs plonk setup, as well as contracts deployment is executed in 6.90 seconds. 
\begin{comment}
    - how is plonk better than the rest, how is plonk weaker than the rest in general?
\end{comment}
The result of the evaluation cycle is described as follows, taking feedback from the recent project workshop into consideration: The implementation of the software artifact zk-DApp represents a seized opportunity to meet the second project requirement of constructing a trade-off between privacy, confidentiality and transparency. This requirement is successfully met, due to the nature of zero-knowledge proofs realizing transparency and confidentiality in a trustless environment of MRO spare parts validation. On the one hand, the use case of MRO KPI validation needs to be revisited and could be replaced to meet more urgent project targets, e.g., other MRO data to be validated. The data captured in the prototype is captured per part identifier only, i.e., the form identifier is obsolete. The time since the last service on the part was performed is relevant as well, and should be added as a third constraint as well (Figure \ref{fig:form}). The data points are to be perceived as stand-alone, i.e., validation calculation checks yet have to be investigated. This also true for any threshold values. Except for the data points itself, further content analysis and feedback in the MRO industry is to be performed in future research to further match the use case in this prototype. One the other hand, the demonstration of the technical functionality is successfull, and the existing DApp for MRO certificate validation and the MRO Smart Hub (in development through Opremic GmbH) can be extended with the zero-knowledge proof functionality of this software artifact. The current code shared with the thesis shows the implementation of most recent feedback of the evaluation cycle. A screenshot of the input screen (in analogy to Figure \ref{fig:form}) is shown below (Figure \ref{fig:form1}).
\begin{figure}[hbt]
	\centering
		\includegraphics[width=1.0\textwidth]{Pictures/bsp2.png}
	\caption{zk-DApp Fill Form}
	\label{fig:form1}
\end{figure}

\begin{comment}
-do one evaluation cycle: which design requirements are met?

feedback from Opremic: 
- MRO data tracked per part Id
- add time since last service as entry, remove form id
- add third constraint
- constraints are stand-alone, no equations yet identified
-add generic attester: Validation Network
- more systematic user interviews needed for this use case, but technical functionality is easily displayed via this use case

- why is the architecture good -->because it tackles the assumptions that are made for the zk-DApp and focuses on another big requirement in the project: the data digitization approach
- comparison of groth16 and plonk performance
- limitations of the implementation
- future outlook with turboplonk

content:
- data has to be made more available: automatic part id verification, automatic assignment of attesters according to fields (e.g. MRO cycles always done by repair shop)
- also automatic threshold identification according to part id and part description: e.g. a turbine needs another threshold than a wing
- data structure will dictate industry effectiveness of zero-knowledge proof implementations in the project as well as adoption
- zk-DApp flexible and easy to adapt to further business cases
- the data structure verification with zero-knowledge proposes another way of making use of ZKP for the requirement to create a industry standard for MRO data digitization and verification


---------------RAPADO Workshop May 8
- certificates must be checked for authenticity first
- then data structure is needed to store part id+serial number+ form number and name of mechanic
- Daten des Mechanikers liegen aber in seiner Organisation, das ist Hauptproblem
- Datenschutz
-current: Stammdatensatz, aus dem statistische evaluierung erfolgen soll
- diese datenstruktu könnte als support system dienen für die validierer, validierung kann bis zu 3 wochen dauern




-am ende unbedingt auf manuellen aufwand eingehen+shortage of validation staff, und die verschwendung weil zweitmarkt fehlt 
- kosten sind am höchsten bei der überfühurng in blockchain-->zkps weg zur kostensenkung
\end{comment}



    \clearpage{\pagestyle{empty}\cleardoublepage}
    \chapter{Conclusion}
\section{Discussion of Results}
\section{Limitations}
\section{Future Work}
    \clearpage{\pagestyle{empty}\cleardoublepage}

% -----------------------------------
\backmatter 
\bibliographystyle{apacite}				% bei natbib in deutsch
\bibliography{./Literature/sources}		% Literaturquellen einbinden 
\newpage
\thispagestyle{empty}

\begin{large}

\vspace*{2cm}

\noindent
I declare that I have authored this thesis independently, that I have not used other than the declared
sources / resources, and that I have explicitly marked all material which has been quoted either
literally or by content from the used sources. 

\vspace{2cm}

\noindent
Berlin, 12 June 2023
\vspace{3cm}

\hspace*{7cm}%
\dotfill\\
\hspace*{8.5cm}%
\textit{(signature)}

\end{large}
 % Eidesstattliche Erklärung (nicht bei Seminararb.)

\end{document}